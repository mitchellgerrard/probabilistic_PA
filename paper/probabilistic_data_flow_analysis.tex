\section{Probabilistic Data Flow Analysis}
\label{sec:pdfa}

\subsection{Meta-comments}

We have chosen to organize the work on prob. data flow analysis based 
on how the probabilistic information is incorporated into the analysis
(e.g., probabilities on data, probabilities on control)
and on the nature of approximation in the analysis, 
i.e., underapproximation, overapproximation, or "tight" approximation.

We plan a separate discussion of how non-deterministic choice is
handled in data flow analysis.

Finally, we plan a brief mention of work that does not fit into 
"basic probabilistic program" category, i.e., programs that use
conditioning.

We would be interested in exposing other dimensions ASAP.  Specifically,
are there different dimensions that might arise due to thinking about 
model checking or symbolic execution?

\subsection{Required Terminology}

The following terms/concepts should be defined earlier in the paper
since we will need them in this section.

\begin{enumerate}

 \item probabilistic program
 \item concrete domain
 \item abstract domain
 \item fixpoint
 \item abstract interpretation/data flow analysis
 \item program trace
 \item path
 \item conditioned distribution
 \item Bayesian inference

\end{enumerate}

We expect that this will be done in the intro and background section.   With
regards to that section it would be ideal if we could have a compact
explanation of non-probabilistic data flow analyis/abstract interpretation,
model checking, and symbolic execution with the attendant concepts.
That will cover most of the above and then we can have a separate
subsection of the background covering the probabilistic 
concepts/terms/definitions.

\subsection{The Outline}

Probabilistic Data Flow Analysis Outline

We are considering approaches that start from classical abstract domains.
and characterize the probability of properties expressed as subsets
of those domains holding at program points.
  - this is equivalent to reasoning about the probability of assertions
    holding or not (in prob sym exe) or probabilistic universal properties
    (in prob model checking) 

\begin{enumerate}

 \item Probabilities on control structure
   \begin{enumerate}
    \item probabilistic information explicitly annotates the control flow
    structure of the program
    \item Frequency analysis
      \begin{enumerate}
       \item Ramalingam
       \item See if we can tie this approach to the linear
  	  operators representing transfer functions
  	  which both Monniaux and Di Pierro use
      \end{enumerate}
   \end{enumerate}

 \item Probabilities on data structure
   \begin{enumerate}
    \item probabilistic information annotates the data structure of the
    program and its influence on control and data is computed through
    the analysis
    \item the rest of the approaches permit probabilistic information to
    be defined as either an environment property (i.e., distribution
    given for an input) or in the type of an expression (i.e., rand call)
   \end{enumerate}

 \item Bounding property probabilities
   \begin{enumerate}
    \item Monniaux
      \begin{enumerate}
       \item From above
       \item From below
      \end{enumerate}
   \end{enumerate}

 \item Estimating property probabilities
   \begin{enumerate}
    \item PAI
     \begin{enumerate}
      \item Di Pierro
      \item "Tight" in a least-square sense
     \end{enumerate}
   \end{enumerate}

 \item Treating uncertainty expressed as non-determinism
   \begin{enumerate}
    \item Monniaux can do this 
   \end{enumerate}

 \item Expanding the scope of analysis
   \begin{enumerate}
    \item ... to different probabilistic properties
      \begin{enumerate}
        \item Chakarov
        \item Fixed Points
        \item Martingales
      \end{enumerate}
    \item ... to more general probabilistic programs
      \begin{enumerate}
        \item Bayesian inference
          \begin{enumerate}
            \item Nori
            \item Modern Probabilistic Programming Languages
          \end{enumerate}
      \end{enumerate}
   \end{enumerate}

\end{enumerate}
