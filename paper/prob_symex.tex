\section{Probabilistic Symbolic Execution}
\label{sec:pse}

\subsection{Introduction}
\begin{itemize}
\item symex produces PCs and interested in SAT/UNSAT and solutions
\item now we are interested in number of solutions, given a profile
\item ISSUE: we assume uniform, but where will non-uniform be described?
\item we can use model counting to count solutions for a PC (given uniform)
\item dividing the solution count by the domain count gives probability of the PC holding
\item call this probabilistic symbolic execution
\end{itemize}

\subsection{Applications}

\begin{itemize}
\item knowing the probability of a PC
\item low-level: probability of a path, reaching a location, returning a value, failing an assertion,... 
\item that can lead to...
\item reliability of a piece of code
\item program understanding
\item program equivalence
\item domain coverage
\end{itemize}

\subsection{Approach}

{\bf following is just placeholder sentences}

Note that the symex algorithm from section blah (background) produces path conditions along the way, which we can then use to do model counting, ...

However, in order to allow us to sample a space of behaviours, i.e. paths, when all paths cannot be exhaustively analysed, we rather use the algorithm below, which samples fully executed symbolic paths. 

\begin{minipage}{0.5\textwidth}
\begin{algorithm}[H]
\renewcommand{\thealgorithm}{}
\floatname{algorithm}{Alg.}
\caption{{\tt pse}$(l,m,\phi)$}
\label{symexe}
\begin{algorithmic}
 \REPEAT
  \STATE $p \gets {\tt symsample}(l_0, m_0, \x{true})$
  \STATE $\x{processPath}(p)$
 \UNTIL {$\x{stoppingSearch}(p)$}
\end{algorithmic}
\end{algorithm}
\end{minipage}
\begin{minipage}{0.5\textwidth}
\begin{algorithm}[H]
\renewcommand{\thealgorithm}{}
\floatname{algorithm}{Alg.}
\caption{{\tt symsample}$(l,m,\phi)$}
\label{symexe}
\begin{algorithmic}
 \IF{$\x{stoppingPath}((\phi)$} 
 \RETURN $\phi$
 \ENDIF
 \WHILE{$\neg branch(l)$}
   \STATE $m \gets m\lrangle{v, e}$
   \STATE $l \gets next(l)$
 \ENDWHILE
 
 \STATE $c \gets (m[\x{cond}(l)]$
 
 \IF{$\x{selectBranch}(c,\phi)$}
   \RETURN {\tt symsample}$(\x{target}(l), m, \phi \wedge c)$
 \ELSE
   \RETURN {\tt symsample}$(\x{next}(l), m, \phi \wedge \neg c)$
 \ENDIF
\end{algorithmic}
\end{algorithm}
\end{minipage}

\subsubsection{stoppingpath}

First explain stoppingPath, since it is arguably also part of symex itself: finite paths not an issue, but infinite paths, i.e. loops on input, must be stopped  {\bf but this needs to move to background really, since the algorithm there is never going to terminate}

\subsubsection{selectBranch}

\begin{itemize}
\item If we sample exhaustively then we could pick branches any which way we want to, but we have to record which ones we have picked to allow termination [ note we will explain this in the processPath section this requires us to explain the counting and subtracting values etc.
\item to do a proper monte carlo simulation we need to consider the counts of each branch
\item could be elaborate here and use heuristics
\item what if the branch is nondeterministic?
\end{itemize}

\subsubsection{stoppingSearch}

\begin{itemize}
\item now we have to consider the domain coverage, i.e. stopping when we have searched a pre-defined level of the domain, this needs to include the discussion about gray paths
\item also need to talk about the statistical approach from FSE paper, hopefully Antonio can do this, the SA contingent is not capable!
\item need to mention sample with replacement here as well
\end{itemize}

\subsubsection{processPath}
\begin{itemize}
\item it checks the property of interest
\item Now we need to discuss the parts about storing the branch and the counts etc. i.e. the non replacement and termination issues
\end{itemize}

\subsection{Nondeterminism}

Here we will not say much but refer to our previous work in the ICSE paper and then the more elaborate approach in the ASE paper.

\subsection{Open Issues}

Can't think of a good name for this section now.

\begin{itemize}
\item Domains for counting: reals, non-linear, structures, strings, \#sat
\item distributed algorithms
\item discussion about input partitions if not going in the next section by Corina
\end{itemize}

