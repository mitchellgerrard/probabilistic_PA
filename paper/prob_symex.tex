\section{Probabilistic Symbolic Execution}
\label{sec:pse}

\subsection{Introduction}
\begin{itemize}
\item symex produces PCs and interested in SAT/UNSAT and solutions
\item now we are interested in number of solutions, given a profile
\item ISSUE: we assume uniform, but where will non-uniform be described?
\item we can use model counting to count solutions for a PC (given uniform)
\item dividing the solution count by the domain count gives probability of the PC holding
\item call this probabilistic symbolic execution
\end{itemize}

\subsection{Applications}

\begin{itemize}
\item knowing the probability of a PC
\item low-level: probability of a path, reaching a location, returning a value, failing an assertion,... 
\item that can lead to...
\item reliability of a piece of code
\item program understanding
\item program equivalence
\item domain coverage
\end{itemize}

\subsection{Approach}

{\bf following is just placeholder sentences}

Note that the symex algorithm from section blah (background) produces path conditions along the way, which we can then use to do model counting, ...

However, in order to allow us to sample a space of behaviours, i.e. paths, when all paths cannot be exhaustively analysed, we rather use the algorithm below, which samples fully executed symbolic paths. 

\begin{minipage}{0.5\textwidth}
\begin{algorithm}[H]
\renewcommand{\thealgorithm}{}
\floatname{algorithm}{Alg.}
\caption{{\tt pse}$(l,m,\phi)$}
\label{symexe}
\begin{algorithmic}
 \REPEAT
  \STATE $p \gets {\tt symsample}(l_0, m_0, \x{true})$
  \STATE $\x{processPath}(p)$
 \UNTIL {$\x{stoppingSearch}(p)$}
\end{algorithmic}
\end{algorithm}
\end{minipage}
\begin{minipage}{0.5\textwidth}
\begin{algorithm}[H]
\renewcommand{\thealgorithm}{}
\floatname{algorithm}{Alg.}
\caption{{\tt symsample}$(l,m,\phi)$}
\label{symexe}
\begin{algorithmic}
 \IF{$\x{stoppingPath}((\phi)$} 
 \RETURN $\phi$
 \ENDIF
 \WHILE{$\neg branch(l)$}
   \STATE $m \gets m\lrangle{v, e}$
   \STATE $l \gets next(l)$
 \ENDWHILE
 
 \STATE $c \gets (m[\x{cond}(l)]$
 
 \IF{$\x{selectBranch}(c,\phi)$}
   \RETURN {\tt symsample}$(\x{target}(l), m, \phi \wedge c)$
 \ELSE
   \RETURN {\tt symsample}$(\x{next}(l), m, \phi \wedge \neg c)$
 \ENDIF
\end{algorithmic}
\end{algorithm}
\end{minipage}

\subsubsection{stoppingpath}

First explain stoppingPath, since it is arguably also part of symex itself: finite paths not an issue, but infinite paths, i.e. loops on input, must be stopped  {\bf but this needs to move to background really, since the algorithm there is never going to terminate}

\subsubsection{selectBranch}

\begin{itemize}
\item If we sample exhaustively then we could pick branches any which way we want to, but we have to record which ones we have picked to allow termination [ note we will explain this in the processPath section this requires us to explain the counting and subtracting values etc.
\item to do a proper monte carlo simulation we need to consider the counts of each branch
\item could be elaborate here and use heuristics
\item what if the branch is nondeterministic?
\end{itemize}

\subsubsection{stoppingSearch}

\begin{itemize}
\item now we have to consider the domain coverage, i.e. stopping when we have searched a pre-defined level of the domain, this needs to include the discussion about gray paths
\item also need to talk about the statistical approach from FSE paper, hopefully Antonio can do this, the SA contingent is not capable!
\item need to mention sample with replacement here as well
\end{itemize}

\subsubsection{processPath}
\begin{itemize}
\item it checks the property of interest
\item Now we need to discuss the parts about storing the branch and the counts etc. i.e. the non replacement and termination issues
\end{itemize}

\subsection{Nondeterminism}

Here we will not say much but refer to our previous work in the ICSE paper and then the more elaborate approach in the ASE paper.

\subsection{Open Issues}

Can't think of a good name for this section now.

\begin{itemize}
\item Domains for counting: reals, non-linear, structures, strings, \#sat
\item distributed algorithms
\item discussion about input partitions if not going in the next section by Corina
\end{itemize}

\subsection{Added by Corina}
\subsection{Symbolic Execution}
Symbolic Execution~\cite{King1976,clarke76:system} is a program
analysis technique that executes programs on unspecified inputs, by
using symbolic inputs instead of concrete data. The state of a
symbolically executed program is defined by the (symbolic) values of
the program variables, a \emph{path condition} ($PC$), and a program
counter. The path condition is a (quantifier-free) boolean formula
over the symbolic inputs; it accumulates constraints on
the inputs to follow that path. The program counter defines the next
statement to be executed. Dynamic symbolic execution (or dynamic test generation or concolic execution)~\cite{godefroid05:dart,Cute,BitBlaze} is a variant of symbolic execution that builds path conditions along concrete program runs and systematically negates and solves collected constraints to drive the execution along different program paths. A key advantage of the technique is the ability of use execution data when symbolic constraints can not be solved. Further the technique has been shown to scale well in practice~\cite{SAGEsummaries}.

A \emph{symbolic execution tree} characterizes the execution paths
followed during symbolic execution. The tree nodes 
represent program states and the arcs the transitions between
states due to the execution of program instructions. Typical applications 
of symbolic execution include test case
generation and error detection, with many tools available
\cite{SPF,godefroid05:dart,tillman-halleux-tap2008,Klee}.
%Symbolic execution of looping programs may result in an infinite
%symbolic execution tree. For this reason, symbolic execution is
%typically run with a (user-specified) bound on the search depth.

Our previous work on probabilistic software analysis used
the symbolic execution tool Symbolic Java PathFinder (SPF)~\cite{SPF}, 
part of the Java PathFinder open-source verification framework. 
However, our algorithms are applicable in the context
of other languages for which symbolic execution tools exists such as
Klee~\cite{Klee} for C. 

\subsection{Probabilistic Software Analysis} 
%Code-level probabilistic analysis is an active area of research with 
%techniques based on abstract interpretation~\cite{Monniaux2000,CousotESOP12}, 
%static analysis~\cite{adje-etal-vstte13,Rajamani1,Rajamani2} and symbolic
%execution~\cite{ISSTA12,PLDI13}.  All these techniques target 
%sequential code and they differ in the type of input
%distributions they consider, the language features supported, and
%the approach used to calculate the number of solutions.
%
We will build on our previous work
from~\cite{geldenhuys-etal-issta2012,filieri-etal-icse2013,Borges2014PLDI}, that handles numeric constraints and
complex data structures as inputs (the latter is not described here,
for brevity). The goal of the analysis is: (1) to identify the 
symbolic constraints characterizing the inputs that make the 
execution satisfy a given property, and then (2) to quantify the 
probability of satisfying the constraints.
For simplicity, we assume the satisfaction of the target property to be 
characterized by the occurrence of a target event (e.g. successful 
termination or failure), but our work extends to bounded 
LTL~\cite{Zuliani2010} as well. 


The analysis works with a limited budget of symbolic paths, obtained with a bounded symbolic execution of the program. 
 Some of these paths lead to
failure and some of them to success (termination without failure). 
These path conditions are classified in two disjoint sets: 
$\textit{PC}^s=\{\textit{PC}_1^s, 
\textit{PC}_2^s, \ldots , \textit{PC}_m^s\}$ and
$\textit{PC}^f=\{\textit{PC}_1^f, 
\textit{PC}_2^f, \ldots , \textit{PC}_p^f\}$.
The path conditions 
may not cover the full input domain due to inherent incompleteness in the analysis (e.g. due to non-terminating loops or non-exhaustive path exploration) -- these remaining paths are called  {\em grey} paths and are used in~\cite{filieri-etal-icse2013} to quantify the confidence one can put in the bounded symbolic analysis.

\subsubsection{Probabilistic Usage Profiles} 

The constraints generated with symbolic execution are analyzed to
quantify the likelihood of an input to satisfy them, where the inputs
are distributed according to given {\em usage
profiles}~\cite{filieri-etal-icse2013}. A usage profile is a probabilistic
characterization of the software interactions with the external world,
e.g. the users or the physical execution environment.  It assigns to
each valid combination of inputs its probability to occur during
 execution. 
 %Usage profiles can come from monitoring the usage of
% actual or similar systems or expert and domain knowledge (physical
% phenomena). 
In~\cite{filieri-etal-icse2013} we assumed that the usage profiles
are given but part of the proposed project will investigate techniques for the 
automated inference of compact usage profiles from the usage data 
available and from the analysis of the program itself.

In~\cite{filieri-etal-icse2013}, we also assumed that all the input variables
range over finite discrete domains, whose joining is generically
indicated as $D$. We relaxed this assumption in more recent work~\cite{Borges2014PLDI}. We profile the expected usage for the program through a profile
$\textit{UP}$, which is a set of pairs $\langle
c_i, p_i \rangle$ where $c_i$ is a \emph{usage scenario} defined as a
(constraint representing a) subset of $D$ and $p_i$ ($p_i\geq 0$) is
the probability that a user input belongs to $c_i$. We further
require, for simplicity, $\{c_i\}$ to be a complete partition of $D$,
and thus $\sum_i p_i=1$. Intuitively, $UP$ is the distribution over
the input space. Notice that $c_i$ could contain even a single element
of $D$, allowing for the finest grained specifications of
$\textit{UP}$.

Given the output of symbolic execution, the probability of success can
be defined as the probability of executing the program ($P$) with
an input satisfying any of the successful path conditions, given the
profile $UP$. This definition can be formalized as
$\textit{Pr}^s(P)=\sum_{i} \textit{Pr}(\textit{PC}_i^s \ | \ \textit{UP})$.
An analogous definition is provided for the probability of failure,
$\textit{Pr}^f(P)$. The probability of grey paths is $1-(\textit{Pr}^s(P)+\textit{Pr}^f(P))$ and it
quantifies the ratio of elements
of the input domain for which neither success nor failure have been
revealed for the current analysis. This information is a measure of
the confidence we can put on the probability estimation, under
the current exploration bound.

\subsubsection{Computing Probabilities using Model Counting} 
To compute the probabilities of path conditions, we use a
quantification procedure for the generated constraints.
In~\cite{filieri-etal-icse2013} we used model counting techniques,
i.e. LattE~\cite{deLoera-etal-2012}, to estimate (algorithmically) the exact
number of points of a bounded (possibly very large) discrete domain
that satisfy linear constraints.  In more recent work~\cite{Borges2014PLDI},  
we developed quantification procedures for
the analysis of programs that have mixed integer and floating point
constraints of arbitrary complexity.

To compute the estimated probability of
success (or failure)  we use the fact that $UP$
defines a partition of the input domain and then, from the law of total
probability~\cite{pestman2009}:
$\textit{Pr}(\textit{PC} \ | \ \textit{UP}) =  \sum_{i} \textit{Pr}(\textit{PC} \ | \ c_i) \cdot p_i$.
Furthermore, from the definition of conditional probability~\cite{pestman2009}:
$\textit{Pr}(\textit{PC} |c_i) = \textit{Pr}(\textit{PC} \wedge c_i) / \textit{Pr}(c_i)$. 

To use model-counting techniques for the computation of the
conditional probabilities, let us define for a constraint $c$ the
function $\sharp(c)$ that returns the number of elements of $D$
satisfying $c$. $\sharp(\cdot)$ is always a finite non negative
integer because we assumed $D$ finite and countable. Under this same
assumption, $\textit{Pr}(c)$ is, by definition~\cite{pestman2009},
$\sharp(c) / \sharp(D)$ (where $\sharp(D)$ is the size of the non-empty
input domain).  Thus, one can express the probability of success as:
\setlength\abovedisplayskip{-1pt}
\setlength\belowdisplayskip{-2pt}
\begin{equation*}
\textit{Pr}^s(P)=\sum_{i} \textit{Pr}(\textit{PC}_i^s \ | \ \textit{UP})=
 \sum_{i} \sum_{j} \textit{Pr}(\textit{PC}_i^s \ | \ c_j) \cdot p_j= \sum_{i} \sum_{j} \frac{\sharp(\textit{PC}_i^s \land c_j)}{\sharp(c_j)} \cdot p_j
\end{equation*}
