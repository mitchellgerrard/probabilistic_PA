%%%%%%%%%%%%%%%%%%%%%%% file typeinst.tex %%%%%%%%%%%%%%%%%%%%%%%%%
%
% This is the LaTeX source for the instructions to authors using
% the LaTeX document class 'llncs.cls' for contributions to
% the Lecture Notes in Computer Sciences series.
% http://www.springer.com/lncs       Springer Heidelberg 2006/05/04
%
% It may be used as a template for your own input - copy it
% to a new file with a new name and use it as the basis
% for your article.
%
% NB: the document class 'llncs' has its own and detailed documentation, see
% ftp://ftp.springer.de/data/pubftp/pub/tex/latex/llncs/latex2e/llncsdoc.pdf
%
%%%%%%%%%%%%%%%%%%%%%%%%%%%%%%%%%%%%%%%%%%%%%%%%%%%%%%%%%%%%%%%%%%%


\documentclass[runningheads,a4paper]{llncs}

\usepackage{amssymb}
\setcounter{tocdepth}{3}
\usepackage{graphicx}

%\usepackage{url}
%\urldef{\mailsa}\path|{dwyer, mgerrard}@cse.unl.edu|    
%\newcommand{\keywords}[1]{\par\addvspace\baselineskip
%\noindent\keywordname\enspace\ignorespaces#1}

\begin{document}

\mainmatter  % start of an individual contribution

% comment macro
\newcommand{\mycomment}[1]{\textit{\textcolor{red}{#1}}}
\newcommand{\ignore}[1]{}

% first the title is needed
\title{Probabilistic Program Analysis}

% a short form should be given in case it is too long for the running head
\titlerunning{Probabilistic Program Analysis}

\author{Matthew B. Dwyer
\and Antonio Filieri\and Jaco Geldenhuys\and Mitchell Gerrard\and\\
Corina Pasareanu\and Willem Visser}
%
\authorrunning{Probabilistic Program Analysis}
% (feature abused for this document to repeat the title also on left hand pages)

% the affiliations are given next; don't give your e-mail address
% unless you accept that it will be published
%\institute{Springer-Verlag, Computer Science Editorial,\\
%Tiergartenstr. 17, 69121 Heidelberg, Germany\\
%\mailsa\\
%\url{http://www.springer.com/lncs}}

%
% NB: a more complex sample for affiliations and the mapping to the
% corresponding authors can be found in the file "llncs.dem"
% (search for the string "\mainmatter" where a contribution starts).
% "llncs.dem" accompanies the document class "llncs.cls".
%

\toctitle{Lecture Notes in Computer Science}
\tocauthor{Authors' Instructions}
\maketitle


\begin{abstract}
The abstract should summarize the contents of the paper and should
contain at least 70 and at most 150 words. It should be written using the
\emph{abstract} environment.
\keywords{}
\end{abstract}

\section{Introduction}
\label{sec:introduction}

Static program analyses aim to calculate properties of 
the possible executions of a program without ever running the program,
and have been an active topic of study for over five decades.
Initially developed to allow compilers to generate more efficient
output programs, by the mid-1970s \cite{fosdick1976data} researchers had
understood that such program analyses could be applied to fault
detection and verification of the absence of specific classes of faults.

The power of these analysis techniques, and what distinguishes them from
simply running a program and observing its behavior, is their
ability to reason about program behavior without knowing all of the
exact details of program execution (e.g., the specific 
input values provided to the program, the set of operating system
thread scheduler decisions).  This tolerance of uncertainty allows analyses
to provide useful information when users don't know exactly how
a program will be used (e.g., when a program is first released, when
embedded systems read sensor inputs from the physical world, or
when it is ported to an operating system with a different scheduler).

Static analyses model uncertainty in program behavior
through the use of various forms of abstraction and symbolic representation.
For example, symbolic expressions are used to encode logical constraints 
in symbolic execution~\cite{king1976symbolic}, to define abstract domains
in data flow analysis~\cite{kildall1973unified,cousot1977abstract}, and to 
capture sets of data values that constitute reachable states via
predicate abstraction~\cite{graf1997predabs}.
Nondeterministic choice is another widely used approach for modeling
uncertainty---for instance, in modeling uncertain branch 
decisions in data flow analysis,
or scheduler decisions in model checking.
While undeniably effective, these approaches sacrifice potentially
important distinctions in program behavior.   

Consider a program that accepts an integer input representing
a person's income.  A static analysis might reason about the program
by allowing any integer value, or, perhaps, by applying
some simple assumption, i.e., that income must be non-negative.
Domain experts have studied income distributions and find that
incomes vary according to a generalized beta distribution 
\cite{mcdonald1984some,thurow1970analyzing}.  Can this type of information be 
exploited to yield useful analysis results when classic
analyses fail, or to reason about new types of program properties?

\ignore{
For decades, there has been a growing awareness of the value of 
incorporating more precise forms of uncertainty into program behavior.  
The field of randomized algorithms has studied how to incorporate
randomness as an additional program input---as a means of achieving
good average case performance, and consequently, as a defense against
intolerable worst case performance.
Programming such algorithms requires that primitives be available
to draw values from probability distributions, and there are many
languages that provide such primitives, e.g., NetLogo \cite{tisue2004netlogo},
the C++ \texttt{$\mathtt{<random>}$} library, and the GNU Scientific library.
}

Whether information about the distribution of
values is embedded within a program or stated as an input assumption,
the semantics of such probabilistic programs is well-understood---and
has long been studied 
\cite{kozen1981semantics,kozen1983probabilistic,jones1990probabilistic,morgan1996probabilistic}
\setcounter{footnote}{0}
\footnote{In recent
years, the term probabilistic program has been generalized beyond
drawing inputs from probability distributions, which we
consider here, to programs that can condition program behavior---by
rejecting certain program runs---and thereby be viewed as
computations over probability distributions.  We refer the reader to the
recent paper by Gordon, Henzinger, Nori and Rajamani \cite{Gordon2014}
for discussion of the analysis of these more general probabilistic programs.}. 
What has lagged behind is the development of frameworks for 
defining and implementing static analysis techniques for such programs.

What would such analyses have to offer?
They can, of course, provide a means of analyzing programs that compute
with values chosen from probability distributions, but they offer much
more.
For example, researchers have explored the use of probabilistic analysis
results to assess the security of software components~\cite{mardziel2013dynamic},
to assess program reliability~\cite{Filieri2013}, to measure program
similarity~\cite{Geldenhuys2012}, 
to characterize the coverage
achieved by an analysis technique~\cite{DwyerASE11}, and to provide information
about program properties when a classic analysis would fail, e.g.,
by running out of memory, time, or due to excessive approximation.

In this paper, we survey work on adapting data flow analysis 
and symbolic execution to use probabilistic information.
We begin with background that provides basic definitions
related to static analysis and probabilistic models.
Section~\ref{sec:overview} attempts to expose some of the key
intuitions and concepts that cross-cut the work in this area.
The following two sections, \ref{sec:pdfa}-\ref{sec:pse}, 
survey work on probabilistic data flow analysis and probabilistic
symbolic execution.  
Section~\ref{sec:computingprobabilities} discusses approaches that
have been developed to reason about the probability of program-related 
events, e.g., executing a path, taking a branch, or reaching a state.
We conclude with a set of open questions
and research challenges that we believe are worth pursuing.


\section{Background and Terminology}
\label{sec:background}


\section{Computing Program Probabilities}
\label{sec:computingprobabilities}

\section{Probabilistic Data Flow Analysis}
\label{sec:pdfa}

\subsection{Meta-comments}

We have chosen to organize the work on prob. data flow analysis based 
on how the probabilistic information is incorporated into the analysis
(e.g., probabilities on data, probabilities on control)
and on the nature of approximation in the analysis, 
i.e., underapproximation, overapproximation, or "tight" approximation.

We plan a separate discussion of how non-deterministic choice is
handled in data flow analysis.

Finally, we plan a brief mention of work that does not fit into 
"basic probabilistic program" category, i.e., programs that use
conditioning.

We would be interested in exposing other dimensions ASAP.  Specifically,
are there different dimensions that might arise due to thinking about 
model checking or symbolic execution?

\subsection{Required Terminology}

The following terms/concepts should be defined earlier in the paper
since we will need them in this section.

\begin{enumerate}

 \item probabilistic program
 \item concrete domain
 \item abstract domain
 \item fixpoint
 \item abstract interpretation/data flow analysis
 \item program trace
 \item path
 \item conditioned distribution
 \item Bayesian inference

\end{enumerate}

We expect that this will be done in the intro and background section.   With
regards to that section it would be ideal if we could have a compact
explanation of non-probabilistic data flow analyis/abstract interpretation,
model checking, and symbolic execution with the attendant concepts.
That will cover most of the above and then we can have a separate
subsection of the background covering the probabilistic 
concepts/terms/definitions.

\subsection{The Outline}

Probabilistic Data Flow Analysis Outline

We are considering approaches that start from classical abstract domains.
and characterize the probability of properties expressed as subsets
of those domains holding at program points.
  - this is equivalent to reasoning about the probability of assertions
    holding or not (in prob sym exe) or probabilistic universal properties
    (in prob model checking) 

\begin{enumerate}

 \item Probabilities on control structure
   \begin{enumerate}
    \item probabilistic information explicitly annotates the control flow
    structure of the program
    \item Frequency analysis
      \begin{enumerate}
       \item Ramalingam
       \item See if we can tie this approach to the linear
  	  operators representing transfer functions
  	  which both Monniaux and Di Pierro use
      \end{enumerate}
   \end{enumerate}

 \item Probabilities on data structure
   \begin{enumerate}
    \item probabilistic information annotates the data structure of the
    program and its influence on control and data is computed through
    the analysis
    \item the rest of the approaches permit probabilistic information to
    be defined as either an environment property (i.e., distribution
    given for an input) or in the type of an expression (i.e., rand call)
   \end{enumerate}

 \item Bounding property probabilities
   \begin{enumerate}
    \item Monniaux
      \begin{enumerate}
       \item From above
       \item From below
      \end{enumerate}
   \end{enumerate}

 \item Estimating property probabilities
   \begin{enumerate}
    \item PAI
     \begin{enumerate}
      \item Di Pierro
      \item "Tight" in a least-square sense
     \end{enumerate}
   \end{enumerate}

 \item Treating uncertainty expressed as non-determinism
   \begin{enumerate}
    \item Monniaux can do this 
   \end{enumerate}

 \item Expanding the scope of analysis
   \begin{enumerate}
    \item ... to different probabilistic properties
      \begin{enumerate}
        \item Chakarov
        \item Fixed Points
        \item Martingales
      \end{enumerate}
    \item ... to more general probabilistic programs
      \begin{enumerate}
        \item Bayesian inference
          \begin{enumerate}
            \item Nori
            \item Modern Probabilistic Programming Languages
          \end{enumerate}
      \end{enumerate}
   \end{enumerate}

\end{enumerate}

\section{Probabilistic Model Checking}
\label{sec:pmc}

\subsection{Required Terminology}

  \begin{enumerate}
   \item discrete time markov chain
   \item markov decision process
   \item model
   \item overapproximation
   \item universal property
   \item counterexample
   \item rewards/costs
   \item safety
   \item trace
   \item abstraction refinement
   \item general probability terms
  \end{enumerate}

\subsection{The Outline}

All based on checking universal properties.

  \begin{enumerate}

   \item Probabilities on control structure
     \begin{enumerate}
      \item probabilistic information explicitly annotates the modeled 
            transition system
      \item discrete time Markov chains
      \item sources of these annotations
        \begin{enumerate}
         \item operational profiles
         \item developer guesswork
        \end{enumerate}
     \end{enumerate}

   \item Bounding property probabilities
     \begin{enumerate}
       \item computing fixed points over some formula
       \item from above/below
       \item extensions with rewards/costs
     \end{enumerate}

   \item Estimating property probabilities (via sampling)
     \begin{enumerate}
      \item difference between probabilistic and statistical model checking
            (``simulating" the model is how we obtain the estimates of this section)
      \item confidence intervals
      \item approximate probabilistic model checking (Herault)
      \item sequential probability ratio test 
     \end{enumerate}

   \item Treating uncertainty expressed as nondeterminism
     \begin{enumerate}
      \item Markov decision processes as extension of DTMCs
      \item minimum/maximum prob. values relating to nondeterminist
            constructs in sym. ex. and in Monniaux's data flow
     \end{enumerate}

   \item Expanding the scope of analysis
     \begin{enumerate}
      \item probabilities on data structures?
        \begin{enumerate}
         \item extract model from actual program and decorate data
               values with probabilities
        \end{enumerate}
      \item run-time probabilistic model checking (Anto)
     \end{enumerate}

  \end{enumerate}

\section{Probabilistic Symbolic Execution}
\label{sec:pse}

\section{Specifying and Inferring Probability Distributions}

\section{Open Questions and Future Directions}


\subsubsection*{Acknowledgments.} The heading should be treated as a
subsubsection heading and should not be assigned a number.

\bibliographystyle{splncs}
\bibliography{ppa}

\end{document}
