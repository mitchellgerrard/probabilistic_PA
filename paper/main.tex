%%%%%%%%%%%%%%%%%%%%%%% file typeinst.tex %%%%%%%%%%%%%%%%%%%%%%%%%
%
% This is the LaTeX source for the instructions to authors using
% the LaTeX document class 'llncs.cls' for contributions to
% the Lecture Notes in Computer Sciences series.
% http://www.springer.com/lncs       Springer Heidelberg 2006/05/04
%
% It may be used as a template for your own input - copy it
% to a new file with a new name and use it as the basis
% for your article.
%
% NB: the document class 'llncs' has its own and detailed documentation, see
% ftp://ftp.springer.de/data/pubftp/pub/tex/latex/llncs/latex2e/llncsdoc.pdf
%
%%%%%%%%%%%%%%%%%%%%%%%%%%%%%%%%%%%%%%%%%%%%%%%%%%%%%%%%%%%%%%%%%%%


\documentclass[runningheads,a4paper]{llncs}

\usepackage{amssymb}
\setcounter{tocdepth}{3}
\usepackage{graphicx}

%\usepackage{url}
%\urldef{\mailsa}\path|{dwyer, mgerrard}@cse.unl.edu|    
%\newcommand{\keywords}[1]{\par\addvspace\baselineskip
%\noindent\keywordname\enspace\ignorespaces#1}

\begin{document}

\mainmatter  % start of an individual contribution

% comment macro
\newcommand{\mycomment}[1]{\textit{\textcolor{red}{#1}}}
\newcommand{\ignore}[1]{}

% first the title is needed
\title{Probabilistic Program Analysis}

% a short form should be given in case it is too long for the running head
\titlerunning{Probabilistic Program Analysis}

\author{Matthew B. Dwyer
\and Antonio Filieri\and Jaco Geldenhuys\and Mitchell Gerrard\and\\
Corina Pasareanu\and Willem Visser}
%
\authorrunning{Probabilistic Program Analysis}
% (feature abused for this document to repeat the title also on left hand pages)

% the affiliations are given next; don't give your e-mail address
% unless you accept that it will be published
%\institute{Springer-Verlag, Computer Science Editorial,\\
%Tiergartenstr. 17, 69121 Heidelberg, Germany\\
%\mailsa\\
%\url{http://www.springer.com/lncs}}

%
% NB: a more complex sample for affiliations and the mapping to the
% corresponding authors can be found in the file "llncs.dem"
% (search for the string "\mainmatter" where a contribution starts).
% "llncs.dem" accompanies the document class "llncs.cls".
%

\toctitle{Lecture Notes in Computer Science}
\tocauthor{Authors' Instructions}
\maketitle


\begin{abstract}
The abstract should summarize the contents of the paper and should
contain at least 70 and at most 150 words. It should be written using the
\emph{abstract} environment.
\keywords{}
\end{abstract}

\section{Introduction}
\label{sec:introduction}

Static program analyses calculate properties of 
the possible executions of a program without ever running the program,
and have been an active topic of study for over five decades.
Initially developed to allow compilers to generate more efficient
output programs, by the mid-1970s \cite{fosdick1976data} researchers 
understood that program analyses could be applied to fault
detection and verification of the absence of specific classes of faults.

The power of these analysis techniques, and what distinguishes them from
simply running a program and observing its behavior, is their
ability to reason about program behavior without knowing all of the
details of program execution (e.g., the specific 
input values provided to the program).
This tolerance of uncertainty allows analyses
to provide useful information when users don't know exactly how
a program will be used.

Static analyses model uncertainty 
through the use of various forms of abstraction and symbolic representation.
For example, symbolic expressions are used to encode logical constraints 
in symbolic execution~\cite{king1976symbolic}, to define abstract domains
in data flow analysis~\cite{kildall1973unified,cousot1977abstract}, and to 
capture sets of data values that constitute reachable states via
predicate abstraction~\cite{graf1997predabs}.
Nondeterministic choice is another widely used approach,
for instance, in modeling branch decisions in data flow analysis.
While undeniably effective, these approaches sacrifice potentially
important distinctions in program behavior.   

Consider a program that accepts an integer input representing
a person's income.  A static analysis might reason about the program
by allowing any integer value, or, perhaps, by applying
some simple assumption, i.e., that income must be non-negative.
Domain experts have studied income distributions and find that
incomes vary according to a generalized beta distribution 
\cite{mcdonald1984some,thurow1970analyzing}.  
With such a distribution the program can now be viewed as a 
\textit{probabilistic program} 
and, beginning with Kozen's seminal work in the early 1980s,
the semantics of such programs has long been studied 
\cite{kozen1981semantics,kozen1983probabilistic,jones1990probabilistic,morgan1996probabilistic}.

For non-probabilistic programs, it was just over six years
from Floyd's foundational work on program semantics~\cite{FLOYD67} 
to Kildall's widely-applicable static analysis framework~\cite{kildall1973unified}.
Sophisticated extensions of Kildall's work are prevalent 
today, e.g., ~\cite{lam11:_soot_java,Lattner:2004:LLVM}, and form 
the basis for modern software development environments.
For probabilistic programs, however, the development of 
static analysis frameworks has taken decades and 
they have not yet reached the level of applicability of 
their non-probabilistic counterparts.

What would such analyses have to offer?
Researchers have explored the use of probabilistic analysis
results to assess the security of software 
components~\cite{mardziel2013dynamic} and to measure side-channel 
leakage~\cite{CSF16,FSE16},
to assess program reliability~\cite{Filieri2013}, to measure program
similarity~\cite{Geldenhuys2012}, 
to characterize fault propagation~\cite{RiskAwareTransformation},
and to characterize the coverage
achieved by an analysis technique~\cite{DwyerASE11}.
\mycomment{Corina: add some more examples here about your latest work}
We believe that there are many more applications for cost-effective
and widely-applicable static analysis frameworks for probabilistic
programs.

In recent years, the term ``probabilistic program'' has been generalized 
beyond Kozen's definition in which programs draw inputs from probability 
distributions.
This more general setting permits the conditioning of program behavior by
allowning certain program runs to be rejected.  
These programs can be viewed as expressing computations over 
probability distributions rather than inputs drawn from a distribution.  
While recent work has just begun to explore the foundations of analysis for
this more general setting~\cite{claret2013bayesian,Gordon2014}, 
in this paper we consider
Kozen's original definition and analysis frameworks targetting such programs.

More specifically, we survey work on adapting data flow analysis 
and symbolic execution to use information about input distributions.
We begin with background that provides basic definitions
related to static analysis and probabilistic models.
Section~\ref{sec:overview} exposes some of the key
intuitions and concepts that cross-cut the work in this area.
The following two sections, \ref{sec:pdfa}-\ref{sec:pse}, 
survey work on probabilistic data flow analysis and probabilistic
symbolic execution.  
While we focus on analysis of imperative programs, we note that
principles exposed in our survey apply to analysis frameworks 
for functional programs as well.
Section~\ref{sec:computingprobabilities} discusses approaches that
have been developed to reason about the probability of program-related 
events, e.g., executing a path, taking a branch, or reaching a state.
We conclude with a set of open questions
and research challenges that we believe are worth pursuing.


\section{Background and Terminology}
\label{sec:background}


\section{Computing Program Probabilities}

\section{Probabilistic Data Flow Analysis}
\label{sec:pdfa}

\subsection{Meta-comments}

We have chosen to organize the work on prob. data flow analysis based 
on how the probabilistic information is incorporated into the analysis
(e.g., probabilities on data, probabilities on control)
and on the nature of approximation in the analysis, 
i.e., underapproximation, overapproximation, or "tight" approximation.

We plan a separate discussion of how non-deterministic choice is
handled in data flow analysis.

Finally, we plan a brief mention of work that does not fit into 
"basic probabilistic program" category, i.e., programs that use
conditioning.

We would be interested in exposing other dimensions ASAP.  Specifically,
are there different dimensions that might arise due to thinking about 
model checking or symbolic execution?

\subsection{Required Terminology}

The following terms/concepts should be defined earlier in the paper
since we will need them in this section.

\begin{enumerate}

 \item probabilistic program
 \item concrete domain
 \item abstract domain
 \item fixpoint
 \item abstract interpretation/data flow analysis
 \item program trace
 \item path
 \item conditioned distribution
 \item Bayesian inference

\end{enumerate}

We expect that this will be done in the intro and background section.   With
regards to that section it would be ideal if we could have a compact
explanation of non-probabilistic data flow analyis/abstract interpretation,
model checking, and symbolic execution with the attendant concepts.
That will cover most of the above and then we can have a separate
subsection of the background covering the probabilistic 
concepts/terms/definitions.

\subsection{The Outline}

Probabilistic Data Flow Analysis Outline

We are considering approaches that start from classical abstract domains.
and characterize the probability of properties expressed as subsets
of those domains holding at program points.
  - this is equivalent to reasoning about the probability of assertions
    holding or not (in prob sym exe) or probabilistic universal properties
    (in prob model checking) 

\begin{enumerate}

 \item Probabilities on control structure
   \begin{enumerate}
    \item probabilistic information explicitly annotates the control flow
    structure of the program
    \item Frequency analysis
      \begin{enumerate}
       \item Ramalingam
       \item See if we can tie this approach to the linear
  	  operators representing transfer functions
  	  which both Monniaux and Di Pierro use
      \end{enumerate}
   \end{enumerate}

 \item Probabilities on data structure
   \begin{enumerate}
    \item probabilistic information annotates the data structure of the
    program and its influence on control and data is computed through
    the analysis
    \item the rest of the approaches permit probabilistic information to
    be defined as either an environment property (i.e., distribution
    given for an input) or in the type of an expression (i.e., rand call)
   \end{enumerate}

 \item Bounding property probabilities
   \begin{enumerate}
    \item Monniaux
      \begin{enumerate}
       \item From above
       \item From below
      \end{enumerate}
   \end{enumerate}

 \item Estimating property probabilities
   \begin{enumerate}
    \item PAI
     \begin{enumerate}
      \item Di Pierro
      \item "Tight" in a least-square sense
     \end{enumerate}
   \end{enumerate}

 \item Treating uncertainty expressed as non-determinism
   \begin{enumerate}
    \item Monniaux can do this 
   \end{enumerate}

 \item Expanding the scope of analysis
   \begin{enumerate}
    \item ... to different probabilistic properties
      \begin{enumerate}
        \item Chakarov
        \item Fixed Points
        \item Martingales
      \end{enumerate}
    \item ... to more general probabilistic programs
      \begin{enumerate}
        \item Bayesian inference
          \begin{enumerate}
            \item Nori
            \item Modern Probabilistic Programming Languages
          \end{enumerate}
      \end{enumerate}
   \end{enumerate}

\end{enumerate}

\section{Probabilistic Model Checking}
\label{sec:pmc}

\section{Probabilistic Symbolic Execution}
\label{sec:pmc}

\section{Specifying and Inferring Probability Distributions}

\section{Open Questions and Future Directions}


\subsubsection*{Acknowledgments.} The heading should be treated as a
subsubsection heading and should not be assigned a number.

\bibliographystyle{splncs}
\bibliography{ppa}

\end{document}
