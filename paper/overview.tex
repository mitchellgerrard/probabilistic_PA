\section{Overview}
\label{sec:overview}

The literature on incorporating probabilistic techniques into 
program analysis is large and growing, technically deep, and quite
varied.  In this paper we cannot hope to cover all of it, but our
intention is to expose key similarities and differences between 
families of approaches and, in doing so, provide the reader with
intuitions that are often missing in the detailed presentation of
techniques, quite helpful in understanding them.

\subsection{Where do the probabilities come from?}
There are two perspectives adopted in the literature.
Programs are implicitly probabilistic because the distributions
from which input values are drawn are not specified in the program,
but are characteristic of the execution environment.
Alternately, programs are explicitly probabilistic in that the statements
within the program define the input probability distributions.
More generally, a program might combine both implicit and explicit
probabilistic calls.  Method \texttt{m} in Figure~\ref{fig:example}
is an example of such a combined probabilistic program.

It is possible to transform explicit probabilistic constructs,
by introducing auxiliary input variables and then specifying
their distributions.   For the example, this would result
in the addition of two integer input variables 
\begin{quote}
\texttt{m(int x, int b1, int b2) \{...} 
\end{quote}
where the two instances of
\texttt{drawBernoulli(0.5)} expressions would be replaced
by \texttt{b1} and \texttt{b2}, respectively.  The input
distribution for these auxiliary inputs would then be specified
as a set of pairs,
\[
\{ (\lambda x.x<0,0), (\lambda x.x=0,0.5), (\lambda x.x=1,0.5), (\lambda x.x>1,0) \}
\]
where the first component defines the characteristic function
of a set of values and the second component defines the probability
of a value in that set.

\subsection{What does the analysis compute?}
There are again two perspectives adopted in the literature.
One can view a probabilistic program as a transformer on probability
distributions and compute the probability distribution, over the
concrete domain, that holds at a program state.
Alternatively, one can view a probabilistic program as a program 
whose inputs happen
to vary in some principled way and compute program properties, 
properties of sets of concrete domain elements, along with a characterization
of how that property varies with varying input.
Within these approaches there are types of approximations
computed for probabilities.  It is common to compute upper bounds
on probabilities for program properties, but lower bounds can 
be computed as well.  In addition, it is possible to estimate the
probability within some margin of error -- an approach that several
techniques explore -- and it is even possible to compute the probability
\textit{exactly} if certain restrictions hold on the program and distributions.

\ignore{
example showing a negation function with drawBernoulli(0.25)
and how it transforms a distribution

show a program that tests whether the absolute value of a number
is greater than 5 where the input is distributed according to N(0,1)
}


\subsection{Handling abstraction expressed through non-determinism}
Any analysis that hopes to scale will invariably be forced to overapproximate
behavior.  As explained earlier in static analyses it is common
to model such overapproximation using non-deterministic choice.
Across all of the analysis techniques we have surveyed, MDPs
have been in these situations since they naturally mix probabilistic
and non-deterministic choice.   
An important consequence of using MDPs is that it is no longer possible
to compute a single probabilistic characterization of a property.
Instead analyses can compute, across the set of all possible sequences
of non-deterministic choice outcomes, the minimal and maximal 
probabilities for a property to hold.

If the minimal probability for a property of interest lies above 
a desired probability threshold, then regardless of how the non-determinism
is resolved the property is guaranteed to hold with the desired 
threshold.  If that is not the case, but the maximal probability for
a property of interest lies above a desired threshold, then there
there is a schedule to resolve the non-determinism that satisfies
the property with at least the desired property.

