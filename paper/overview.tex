\section{Overview}
\label{sec:overview}

The literature on incorporating probabilistic techniques into 
program analysis is large and growing, technically deep, and quite
varied.  In this paper we cannot hope to cover all of it, but our
intention is to expose key similarities and differences between 
families of approaches and, in doing so, provide the reader with
intuitions that are often missing in the detailed presentation of
techniques, quite helpful in understanding them.

\subsection{Where do the probabilities come from?}

\subsection{What does the analysis compute?}
There are two perspective adopted in the literature.
(1) One can view a probabilistic program as a transformer on probability
distributions and compute the probability distribution, over the
concrete domain, that holds at a program state.
(2) View a probabilistic program as a program whose inputs happen
to vary in some principled way and compute program properties, 
properties of sets of concrete domain elements, along with a characterization
of how that property varies with varying input.

\ignore{
example showing a negation function with drawBernoulli(0.25)
and how it transforms a distribution

show a program that tests whether the absolute value of a number
is greater than 5 where the input is distributed according to N(0,1)
}

Within these there are different approaches taken to estimating
these quantities.  Upper bounds, lower bounds, expectations, ...

\subsection{Handling abstraction expressed through non-determinism}
This cross-cutting theme is addressed in all program analyses that
hope to scale. 

All of the techniques use MDPs and then analyze maximal/minimal
schedules.


