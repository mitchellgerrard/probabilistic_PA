\section{Computing Program Probabilities}
\label{sec:computingprobabilities}

\mycomment{Matt: here we need to cover things like sampling approaches
to estimate probabilities, model counting, hybrid approaches that mix these}

\mycomment{Matt: we should pull the discussion of counting here from
the prob. symbolic execution section}

\mycomment{Matt: it would be good if there was some discussion about
how to handle a given distribution.  For example, how would prob
sym exe handle a call to a function to return a value from N(0,1)
(a normal distribution with mean 1 and standard distrubution 0).
Equivalently how would this be specified as a usage profile.
Is there something better than relying on a person to write this down?
I know there is tons of work published on this, but is there a simple
approach we can describe or that you've used.}


\begin{itemize}
	\item Goal: given a finite set of variables $V=\{v_1, v_2, \dots, v_n\}$, where $v_i$ has domain $d_i$ and comes with a probability distribution $\mathcal{P}_i: d_i \to [0, 1]$, and given a constraint $\phi: V \to \{true, false\}$, compute the probability of $\phi$ being satisfied (i.e., return true).
\end{itemize}

\subsection{Exact and numerical approaches}
	\begin{enumerate}
		\item Finite domains
			\begin{enumerate}
				\item Linear integers (Latte, Barvinok, Omega for negation; used in our works)
				\item Strings (bounds: \url{http://www.comp.nus.edu.sg/~shweta24/publications/smc\_pldi14.pdf} ; automata-based exact and upperbounds: \url{http://www.cs.ucsb.edu/~bultan/publications/model-counting.pdf}; we should also check the related work thereof)
				\item Data structures (icse13 and spin15)
				\item Sat and smt (\url{http://arxiv.org/pdf/1306.5726v3.pdf}, \url{http://arxiv.org/pdf/1411.0659.pdf}; these papers should have been published, we should check related work in there)
			\end{enumerate}
		
		\item Uncountable domains
			\begin{enumerate}
				\item Symbolic and numerical integration (interesting, but symbolic does not scale apart from a few simple cases, while numeric suffers for large cardinality of input domains; however: general, available off-the-shelf, parallelizable for numerical)
			\end{enumerate}
	\end{enumerate}


\subsection{Sampling-based approaches}
	\begin{itemize}
		\item Sampling from uniform distributions (base case, best we can if no input distribution is available; based on classic probability, i.e., count success over total)
		\item Discretization of non-uniform distributions (useful when a finite number of usage scenarios are available; can approximate any distribution with arbitrary accuracy; most straightforward when inferring distributions from past executions, i.e., histograms; it does not scale for fine-grained approximations)
		\item Distribution-aware sampling (quite straightforward for distributions over numerical domains; requires more complex, unbiased, input generation techniques when sampling from other domains, e.g., data structures, but similar to random testing; scalability issues when sampling correlated input variables)
      
		\item Monte Carlo methods
			\begin{itemize}
				\item Hit or miss Monte Carlo
				\item \mycomment{Anto: Mention Crude montecarlo for integration? Gibbs and MCMC sampling will be just mentioned; especially MCMC is used by the MSR guys}
				\item Frequentist and Bayesian estimators (used in prob. model check.)
					\begin{itemize}
						\item historically frequentist first, with static bounds (based on Chernoff's) for prob MC; then sequential ratio tests, still frequentist.
						\item Bayesian from CMU
					\end{itemize}
				\item The variance issue and convergence acceleration techniques (just a paragraph with some refs):
					\begin{itemize}
						\item Importance sampling (used in prob. model check.)
						\item Importance slicing (used in prob. model check.)
					\end{itemize}

				\item Exploiting he law of total probability for composing different estimators: Interval constraints propagation (PLDI14) and statistical/exact (FSE14)
				\item Dealing with nondeterminism (freq and bayesian in prob MC, ASE14)	
			\end{itemize}
			
			\item approximate $\sharp$-sat and $\sharp$-smt (\url{http://arxiv.org/pdf/1306.5726v3.pdf}, )


		 
	\end{itemize}
