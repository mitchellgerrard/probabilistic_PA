\section{Probabilistic Data Flow Analysis}
\label{sec:pdfa}

As we will see in all sections, probabilistic data flow
analysis moves from the true/false nature of its classical
counterpart to the probably-true/probably-false nature of
a static analysis extended with probabilities.
The shift from the qualitative to the quantitative allows
you to incorporate probabilistic information into the
analysis in different ways.

\subsection{Probabilities on Control Structure}

Initial work in extending data flow analysis 
techniques with probabilities did not consider
the semantics of the program; instead, already-given probabilities 
were attached to nodes in that program's control flow graph.  
The goal was to predict the probability of an expression evaluating 
to some value or type at runtime, which could 
allow you to perform useful program optimizations.

This approach begins with a control flow graph where each edge $e$ is 
mapped to the probability that $e$ is taken during execution.
The sum of all probabilities leaving any control flow node must be 1
(excepting the exit node).
These probabilities may be obtained through heuristics, profiling,
or some static analysis.
Imagine an execution trace following some path along the edges of
the control flow graph.
The probability of executing that trace is expressed as the product of 
edge probabilities along this path.
So the probability of executing some program point can be seen as the
summation of the probabilities of traces which can reach that program
point.

Within this bag of execution traces which reach point $u$, we want to 
find the portion of traces which satisfy some data flow fact $d$.  
The ratio of the satisfying portion to the size of the trace bags
gives us the probability of fact $d$ holding at point $u$.

{\sl 
Fragments. 

Kildall's general framework for computing fixpoints
over this annotated flow graph.

This assumes execution history does not matter (analysis is path 
insensitive).
Later work (Mehofer) adds some path sensitivity, but as Ramalingam's
framework deals with exploded control flow graphs, a fully 
path-sensitive approach is not tractable.
}

\subsection{Probabilities on Data Structure}

Within the last 15 years, research in probabilistic data flow analysis
began incorporating probabilistic information directly into
the semantics of a program.
This is typically done using a variation on Kozen's 
probabilistic semantics alongside traditional data
flow techniques.
Embedding probabilities into the semantics allows probabilistic
information to influence how both control {\sl and} data structure
probabilities are computed during the analysis.

The remaining approaches in this section permit probabilistic
information to be defined as either an environment property
(i.e., distribution given for an input) or in the type of
an expression (i.e., rand call).
Many of these techniques use the framework of abstract
interpretation.
We will deal primarily with classical
abstract domains, where probabilities of properties are
expressed as subsets of those domains holding at program
points.

\subsection{Bounding Property Probabilities}

There are many ways to measure properties along some probability
in a data flow analysis.
One way to measure is by using upper or lower bounds, e.g. {\sl the 
probability of violating $\phi$ is less than $0.0001\%$};

\subsection{Estimating Property Probabilities}

Another way to measure properties along some probability is to
estimate some ``tight" approximation, e.g.
{\sl this variable will be an even integer at this program point
$67\%$ of the time}.

\subsection{Modeling Nondeterminism}

Monniaux recognizes that there are variations of uncertainty,
and that not every program property should be modeled by a
probability distribution.
For instance, a user may exploit an unlikely control sequence
in a vending machine to get free candy bars. 
If word gets out, the probability of
this exploited behavior occurring is poorly modeled by a uniform 
random distribution.
It is better to treat this kind of input nondeterminisitically.

In one of Monniaux's semantics, variables that can be tied to a known
distribution are cleanly separated from those that cannot.
Assignments to variables drawn from a distribution are reasoned about
through numerical sampling (Monte Carlo), while nondeterminisic choices 
are explored using abstract interpretation.
This combination of methods provides upper bounds on the probability
of outcomes, where one domain is associated with a distribution 
and the other (nondeterministic) domain is modeled using worse-case 
behavior.

\subsection{Expanding the Scope}

\subsection{Meta-comments}

We have chosen to organize the work on prob. data flow analysis based 
on how the probabilistic information is incorporated into the analysis
(e.g., probabilities on data, probabilities on control)
and on the nature of approximation in the analysis, 
i.e., underapproximation, overapproximation, or "tight" approximation.

We plan a separate discussion of how non-deterministic choice is
handled in data flow analysis.

Finally, we plan a brief mention of work that does not fit into 
"basic probabilistic program" category, i.e., programs that use
conditioning.

We would be interested in exposing other dimensions ASAP.  Specifically,
are there different dimensions that might arise due to thinking about 
model checking or symbolic execution?

\subsection{Required Terminology}

The following terms/concepts should be defined earlier in the paper
since we will need them in this section.

\begin{enumerate}

 \item probabilistic program
 \item concrete domain
 \item abstract domain
 \item fixpoint
 \item abstract interpretation/data flow analysis
 \item program trace
 \item path
 \item conditioned distribution
 \item Bayesian inference

\end{enumerate}

We expect that this will be done in the intro and background section.   With
regards to that section it would be ideal if we could have a compact
explanation of non-probabilistic data flow analyis/abstract interpretation,
model checking, and symbolic execution with the attendant concepts.
That will cover most of the above and then we can have a separate
subsection of the background covering the probabilistic 
concepts/terms/definitions.

\subsection{The Outline}

Probabilistic Data Flow Analysis Outline

We are considering approaches that start from classical abstract domains.
and characterize the probability of properties expressed as subsets
of those domains holding at program points.
  - this is equivalent to reasoning about the probability of assertions
    holding or not (in prob sym exe) or probabilistic universal properties
    (in prob model checking) 

\begin{enumerate}

 \item Probabilities on control structure
   \begin{enumerate}
    \item probabilistic information explicitly annotates the control flow
    structure of the program
    \item Frequency analysis
      \begin{enumerate}
       \item Ramalingam
       \item See if we can tie this approach to the linear
  	  operators representing transfer functions
  	  which both Monniaux and Di Pierro use
      \end{enumerate}
   \end{enumerate}

 \item Probabilities on data structure
   \begin{enumerate}
    \item probabilistic information annotates the data structure of the
    program and its influence on control and data is computed through
    the analysis
    \item the rest of the approaches permit probabilistic information to
    be defined as either an environment property (i.e., distribution
    given for an input) or in the type of an expression (i.e., rand call)
   \end{enumerate}

 \item Bounding property probabilities
   \begin{enumerate}
    \item Monniaux
      \begin{enumerate}
       \item From above
       \item From below
      \end{enumerate}
   \end{enumerate}

 \item Estimating property probabilities
   \begin{enumerate}
    \item PAI
     \begin{enumerate}
      \item Di Pierro
      \item "Tight" in a least-square sense
     \end{enumerate}
   \end{enumerate}

 \item Treating uncertainty expressed as non-determinism
   \begin{enumerate}
    \item Monniaux can do this 
   \end{enumerate}

 \item Expanding the scope of analysis
   \begin{enumerate}
    \item ... to different probabilistic properties
      \begin{enumerate}
        \item Chakarov
        \item Fixed Points
        \item Martingales
      \end{enumerate}
    \item ... to more general probabilistic programs
      \begin{enumerate}
        \item Bayesian inference
          \begin{enumerate}
            \item Nori
            \item Modern Probabilistic Programming Languages
          \end{enumerate}
      \end{enumerate}
   \end{enumerate}

\end{enumerate}
