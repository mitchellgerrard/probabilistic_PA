\section{Scope and Background}
\label{sec:background}

\ignore{
\begin{enumerate}
\item abstraction refinement
\item fixpoint
\end{enumerate}
}

We focus in this paper on programs that draw input variables
from given probability distributions, or equivalently that make
calls on functions returning values drawn from given distributions
such as those provided by the C++ \texttt{<random>} library.
\mycomment{We should pull an example from one of the later sections
back here to illustrate the idea.}

While researchers have developed analyses that consider a wide
range of program properties, in this survey
we restrict our attention to program
properties that can be encoded as boolean predicates that
can be embedded in the program,
e.g., \texttt{assert} statements, to simplify the explanation
of how probabilities are incorporated into the analyses.
These are refered to as \textit{invariant properties} since
they are intended to hold at every state reached
in all program program executions

\subsection{Programs and Program Analyses}
A program defines a set of execution \textit{traces} each of
which is a sequence of \textit{concrete states}, i.e., 
the current program counter and a map from memory locations to values.
A program \textit{satisfies} an invariant property if in all states in
all traces the predicate evaluates to true, otherwise the program
\textit{falsifies} the property.

A key concept in the program analysis frameworks we survey is
\textit{symbolic abstraction}.  A symbolic abstraction is a 
representation of a set of states.  Abstractions can be encoded
in a variety of forms, e.g., logical formula or binary
decision diagrams \cite{BDD}.  

Analyses that seek to prove the satisfaction of properties generally
define abstractions that \textit{overapproximate} the set of program
states, whereas that seek to falsify properties generally define
abstractions that \textit{underapproximate} the set of program states.

With overapproximating analyses it is common to define an \textit{abstract
domain}, $\mathcal{A}$, 
which symbolically represents a set of concrete states which 
are said to be defined over the \textit{concrete domain}, $\mathcal{A}$.
For any reachable state of a program a pair of abstraction 
and concretization functions, 
$\alpha : \mathcal{C} \mapsto \mathcal{A}$ and  
$\gamma : \mathcal{A} \mapsto 2^\mathcal{C}$,
serve to relate the concrete and abstract domains such that 
$\forall c \in \mathcal{C} : \alpha(c) \in \mathcal{A}$ and $c \in \gamma(\alpha(c))$.
Such an abstract domain is typically partially ordered, $\sqsupseteq$,
such that 
$\forall a,a' \in \mathcal{A} :  a \sqsupseteq a' \implies \gamma(a) \supseteq \gamma(a')$.

Data flow analysis (and abstract interpretation) can be viewed as a
non-standard interpretation of program executions over an abstract
domain.  The semantics of program statements is lifted to operate
on a set of states, encoded as an element of the abstract domain,
rather than a single concrete state.  For a program statement $\tau$,
$\tau^\#$ defines its abstract semantics such that
$\forall c, c' \in \mathcal{C} : \tau(c) = c' \implies \tau^\#(\alpha(c)) \sqsupseteq \alpha(c')$.  This implies the classic overapproximating correctness
relation for abstracted program statements:
$\tau^\# \sqsupseteq \alpha \circ \tau \circ \gamma$.

\mycomment{Matt: add definitions of underapproximation and whatever
else symbolic execution needs}

\mycomment{Matt: develop a figure which shows pseudo code for model checking,
data flow analysis, and symbolic execution side by side}

\mycomment{Matt: walk through an explanation of that figure which
is where we will define what a fixpoint is}

\noindent\begin{minipage}[t]{0.32\textwidth}
\begin{algorithm}[H]
\renewcommand{\thealgorithm}{}
\floatname{algorithm}{Alg.}
\caption{{\tt mc}$(l,m_0)$}
\label{mc}
\begin{algorithmic}
 \WHILE{$\neg branch(l)$}
   \STATE $m \gets m\lrangle{v, e}$
   \STATE $l \gets next(l)$
 \ENDWHILE
\end{algorithmic}
\end{algorithm}
\end{minipage}%
\hfill
\begin{minipage}[t]{0.32\textwidth}
\begin{algorithm}[H]
\renewcommand{\thealgorithm}{}
\floatname{algorithm}{Alg.}
\caption{{\tt dfa}$(l,m_0)$}
\label{mc}
\begin{algorithmic}
 \WHILE{$\neg branch(l)$}
   \STATE $m \gets m\lrangle{v, e}$
   \STATE $l \gets next(l)$
 \ENDWHILE
\end{algorithmic}
\end{algorithm}
\end{minipage}%
\hfill
\begin{minipage}[t]{0.32\textwidth}
\begin{algorithm}[H]
\renewcommand{\thealgorithm}{}
\floatname{algorithm}{Alg.}
\caption{{\tt symx}$(l,m,\phi)$}
\label{symexe}
\begin{algorithmic}
 \WHILE{$\neg branch(l)$}
   \STATE $m \gets m\lrangle{v, e}$
   \STATE $l \gets next(l)$
 \ENDWHILE

 \STATE $c \gets m[\x{cond}(l)]$

 \IF{SAT$(\phi \wedge c)$}
   \STATE {\tt symx}$(\x{target}(l), m, \phi \wedge c)$
 \ENDIF

 \IF{SAT$(\phi \wedge \neg c)$}
   \STATE {\tt symx}$(\x{next}(l), m, \phi \wedge \neg c)$
 \ENDIF
\end{algorithmic}
\end{algorithm}
\end{minipage}

\subsection{Probabilities and Probabilistic Models}
Define and explain the following
\begin{itemize}
\item probability measure
\item rewards/costs
\item discrete time Markov chain
\item discrete time Markov decision process
\end{itemize}

