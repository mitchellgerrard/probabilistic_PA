\section{Scope and Background}
\label{sec:background}

Here are some requirements from the following section outlines (more to come):

From model checking:
  \begin{enumerate}
   \item discrete time markov chain
   \item markov decision process
   \item model
   \item overapproximation
   \item universal property
   \item counterexample
   \item rewards/costs
   \item safety
   \item trace
   \item abstraction refinement
   \item general probability terms
  \end{enumerate}

From data flow analysis:
\begin{enumerate}
 \item probabilistic program
 \item concrete domain
 \item abstract domain
 \item fixpoint
 \item abstract interpretation/data flow analysis
 \item program trace
 \item path
 \item conditioned distribution
 \item Bayesian inference
\end{enumerate}

From symbolic execution:
\begin{enumerate}
 \item TBD
\end{enumerate}



Thoughts on this section:
\begin{itemize}
\item we need to establish the scope/limits of the paper; our goal is to give people a sense of how probabilistic reasoning has been incorporated into 3 program analysis frameworks
\item we want to introduce terminology and definitions pretty concisely here
\item if we decide to focus on explicit state probabilistic model checking, then I think we could pretty easily have 3 little algorithms in this section that showed non-probabilistic versions of model checking, data flow analysis, and symbolic execution
\item these algorithms could use very similar notation, e.g., state, transition, path, and we could discuss notions of state abstraction and the like with reference to that terminology/notation
\item we will need a separate subsection on probability and it should provide a few basic definitions and a quick discussion of frequentist and Bayesian approaches.
\end{itemize}
