\subsection{Probabilistic Data Flow Analysis}
\label{sec:pdfa}

The key challenge in probabilistic data flow analysis is
determining how probabilities are incorporated into the control
and data abstractions upon which it is based.
As in the classical case, we will see that probabilistic data flow analysis
can be thought of as model checking
of probabilistic abstract interpretations, and in some
cases, the model checking itself must be made probabilistic.

\subsubsection{Control Flow Probabilities}
Early work in extending data flow analysis 
techniques with probabilities did not consider
the influence of control and data flow on probabilities.
Instead, user-defined probabilities 
were attached to nodes in the program's control flow graph.  
This allowed the analysis to estimate
the probability of an expression evaluating 
to some value or type at runtime, which was used to enable
program optimization.

This approach begins with a control flow graph where each edge is 
mapped to the probability that it is taken during execution.
The left side of Figure~\ref{fig:pcfgs} shows the probabilistic CFG for the
example of Figure~\ref{fig:example}, given that input $x$
is uniformly distributed in the range $[1,100]$.  
This program is simple enough that the branch probabilities can
be easily computed---importantly the probabilities for each
branch along a path are independent.

The sum of all probabilities leaving any control flow node must be 1
(except for the exit node).
The probability of executing a path is the product of 
edge probabilities along the path.
Thus, the probability of reaching a program state is the
sum of the probabilities of traces that reach that state.

\begin{figure}[t]
\begin{minipage}[b]{0.5\linewidth}
\tikzstyle{every node}=[text centered, font=\scriptsize, inner sep=1pt]
\begin{tikzpicture}[scale=0.5, level/.style={sibling distance = 30mm, level distance = 15mm}]
  \node[name=entry, xshift=0mm, yshift=20mm] {\texttt{2 if(drawB...}}
   child{
    node[xshift=-5mm] {\texttt{3 if(drawB...}}
     child{
      node[xshift=-3mm] {\texttt{4 if(x <= 60)}}
       child{
        node[name=e1,yshift=-2mm,xshift=3mm] {\texttt{5 ...}}
        edge from parent[->] node[left,xshift=-1mm] {0.6}
       }
       child{
        node[xshift=-3mm, align=left] {\texttt{7 assert}\\
                           \texttt{~~false}}
        edge from parent[->] node[right,xshift=1mm] {0.4}
       }
      edge from parent[->] node[left,xshift=-1mm] {0.5}
     }
     child{
      node[xshift=3mm] {\texttt{9 if(x <= 30)}}
       child{
        node[name=e2,yshift=-2mm,xshift=3mm] {\texttt{10 ...}}
        edge from parent[->] node[left,xshift=-1mm] {0.3}
       }
       child{
        node[name=m1,xshift=-3mm, align=left] {\texttt{12 assert}\\
                                               \texttt{~~false}}
        edge from parent[->] node[right,xshift=1mm] {0.7}
       }
      edge from parent[->] node[right,xshift=1mm] {0.5}
     }
     edge from parent[->] node[left,xshift=-1mm] {0.5}
   }
   child{
    node[xshift=10mm] {\texttt{15 if(x <= 55)}}
     child{
      node[name=e3,yshift=-2mm,xshift=3mm] {\texttt{16 ...}}
      edge from parent[->] node[left,xshift=-1mm] {0.55}
     }
     child{
      node[xshift=-3mm,align=left] {\texttt{18 assert}\\
                                    \texttt{~~false}}
      edge from parent[->] node[right,xshift=1mm] {0.45}
     }
    edge from parent[->] node[right,xshift=1mm] {0.5}
   }
;
\node[name=r,below of=entry, yshift=-30mm] {\texttt{20 \}}};
\draw[->] (e1.south) to ($(r.north)+(-3mm,0)$);
\draw[->] (e2.south) to ($(r.north)+(0,0)$);
\draw[->] (e3.south) to ($(m1.east)+(5mm,0)$) to ($(r.north)+(3mm,0)$);
\end{tikzpicture}
\end{minipage}%
\begin{minipage}[b]{0.5\linewidth}
\tikzstyle{every node}=[text centered, font=\scriptsize, inner sep=1pt]
\begin{tikzpicture}[scale=0.5, level/.style={sibling distance = 30mm, level distance = 15mm}]
  \node[name=entry, xshift=0mm, yshift=20mm] {\texttt{2 if(unknown...}}
   child{
    node[xshift=-5mm] {\texttt{3 if(unknown...}}
     child{
      node[xshift=-3mm] {\texttt{4 if(x <= 60)}}
       child{
        node[name=e1,yshift=-2mm,xshift=3mm] {\texttt{5 ...}}
        edge from parent[->] node[left,xshift=-1mm] {0.6}
       }
       child{
        node[xshift=-3mm,align=left] {\texttt{7 assert}\\
                                      \texttt{~~false}}
        edge from parent[->] node[right,xshift=1mm] {0.4}
       }
      edge from parent[->] node[left,xshift=-2mm] {?$_2$}
     }
     child{
      node[xshift=3mm] {\texttt{9 if(x <= 30)}}
       child{
        node[name=e2,yshift=-2mm,xshift=3mm] {\texttt{10 ...}}
        edge from parent[->] node[left,xshift=-1mm] {0.3}
       }
       child{
        node[name=m1,xshift=-3mm,align=left] {\texttt{12 assert}\\
                                              \texttt{~~false}}
        edge from parent[->] node[right,xshift=1mm] {0.7}
       }
      edge from parent[->] node[right,xshift=2mm] {1-?$_2$}
     }
     edge from parent[->] node[left,xshift=-2mm] {?$_1$}
   }
   child{
    node[xshift=10mm] {\texttt{15 if(x <= 55)}}
     child{
      node[name=e3,yshift=-2mm,xshift=3mm] {\texttt{16 ...}}
      edge from parent[->] node[left,xshift=-1mm] {0.55}
     }
     child{
      node[xshift=-3mm,align=left] {\texttt{18 assert}\\
                                    \texttt{~~false}}
      edge from parent[->] node[right,xshift=1mm] {0.45}
     }
    edge from parent[->] node[right,xshift=2mm] {1-?$_1$}
   }
;
\node[name=r,below of=entry, yshift=-30mm] {\texttt{20 \}}};
\draw[->] (e1.south) to ($(r.north)+(-3mm,0)$);
\draw[->] (e2.south) to ($(r.north)+(0,0)$);
\draw[->] (e3.south) to ($(m1.east)+(5mm,0)$) to ($(r.north)+(3mm,0)$);
\end{tikzpicture}
\end{minipage}
\caption{Probabilistic CFG (left) and MDP (right)}
\label{fig:pcfgs}
\end{figure}



To compute the probability of a data flow fact holding 
at a program point, Ramalingam uses a slightly
modified version of Kildall's dataflow analysis framework
~\cite{ramalingam1996data}.
Instead of the usual semilattice with an idempotent meet
operation, a non-idempotent addition operator is used.
The usual properties of the meet operation can be
relaxed, because instead of computing an invariant dataflow
fact, we only want the summation of probabilities of all
traces reaching a certain point.
The expected frequencies may now be computed as the least
fixpoint using the traditional iterative data flow algorithm;
the quantity becomes a
{\sl sum-over-all-paths} instead of a {\sl meet-over-all-paths}.

Ramalingam's sum-over-all-paths approach is reminiscent of
the approach taken in probabilistic model checking of Markov chains.
In the latter approach, a system of equations is formulated
whose solution yields the probability of some property
holding---so-called \textit{quantitative} properties in 
PRISM \cite{kwiatkowska2010advances}.   Ramalingam's analysis
effectively solves an equivalent system of equations.  

These techniques rely on being able to annotate
branch decisions in the program (or model, in PRISM's case)
with probabilities.  When the input distributions governing
branches is unknown or when the branches along a path are 
dependent the techniques described above cannot be applied.
The discussions below on non-determinism and in Sections~\ref{sec:pse} 
and \ref{sec:computingprobabilities} describe ways
to address this problem.

\subsubsection{Abstract Data Probabilities}
Within the last 15 years, researchers 
began incorporating probabilistic information directly into
the semantics of a program and then abstracting over 
those semantics 
\cite{monniaux2000abstract,smith2008probabilistic,cousot2012probabilistic}
to enable data flow analysis.
This is typically done using a variation on Kozen's 
probabilistic semantics \cite{kozen1981semantics} 
alongside abstract interpretation and data flow techniques.
Embedding probabilities into the semantics allows 
both control flow and data values to influence
the property probabilities computed during the analysis.

\paragraph{Abstracting Probability Distributions}
The pioneering work in this area, by 
Monniaux \cite{monniaux2000abstract,monniaux2001backwards},
developed the key insights that other work has built on. 
The goal is to exploit the rich body of work on developing
abstract domains and associated transformers, and to extend this work 
so as to record bounds on probability measures for the concrete values
described by domain elements.

Monniaux's work takes the view that probabilistic programs 
effectively transform an input distribution into an output
distribution.  More generally, probabilistic programs compute 
a distribution that characterizes each state in the program.   
He develops a probabilistic abstract domain, $\mathcal{A}_p$, 
as a collection of pairs, $\mathcal{A} \times [0,1]$.
The intuition is that a classic abstract domain is paired
with a \textit{bounding probability weight} that is used
to compute an upper bound on the values mapped by that domain.
Consider a probabilistic abstract state, $pa \in \mathcal{A} \times [0,1]$,
an upper approximation of the probability of a value $v$
in that state is given by
$Pr(v) \le \sum_{(a,w) \in pa \wedge c \in \gamma(a)} w$,
where $\gamma$ is maps a symbolic abstraction to
the set of values it describes.

In Monniaux's work, multiple abstract domain elements can map onto a given
concrete value; each of these abstract domain elements'
weights must be totalled to bound the probability of
the given concrete value.
As an example, let $\mathcal{A}$ be the interval abstraction
applied to a single integer value
and let $pa = \{([1,5],0.1),([3,7],0.1),\ldots\}$.  
For a value of $2$, only the first pair would apply, since
$2 \not\in [3,7]$, contributing $0.1$ to the bound on $Pr(2)$.
For a value of $3$, both pairs would apply and contribute
their sum of $0.2$ to the bound on $Pr(3)$.

To clarify, these weight components are \textit{not} bounds on the probability
of the abstract domain as a whole, but rather are bounds on the probability
of each concrete element represented by the abstract domain.
This simplifies the formulation of the probabilistic abstract
transformers, e.g., the extension of $+^\#$
to account for $\mathcal{A}_p$, but it means that additional
work is required to compute the probability of a property holding.
In essence, this requires estimating the size of the concretization
of the abstract domain element and then multiplying by the computed bound for
each concrete value.  

It is important to note that an upper or lower
bound on a probability distribution
is not itself a distribution, since the sum across the domain
may be greater than $1$.
This poses challenges for modular probabilistic data flow analyses.

We will see that the techniques from 
Section~\ref{sec:computingprobabilities} can be applied to
the problem of counting the concretization of an abstract domain element 
that is encoded as a logical formula.  This may offer a potential
connection between data flow analyses formulated over distributions
and those formulated over abstract states---which we discuss below.

The design of probabilistic abstract transformers can be subtle.
For statements that generate variables drawn from a probability
distribution, an upper approximation of the distribution for
regions of the abstract domain is required.  
For sequential statements, weight components are propogated
and abstract domain elements are updated by the underlying transformer.

For conditionals, the transformer can be understood
as filtering the abstract domain between those execution environments which
satisfy the conditional and those which falsify the conditional. 
The probabilistic abstract transformer need only apply
that filter to the first component of
the tuple (the elements of the underlying abstract domain), 
leaving the weight unchanged.
For instance, consider the abstract domain of an interval of 
integers defined by the tuple, $([-5,5],0.1)$. 
If this domain holds before a conditional 
{\tt if(x<0)\{...\}}, then applying the filter on the true branch
results in $([-5,-1],0.1)$ and 
applying the filter on the false branch results in $([0,5],0.1)$.
The space is reduced; the weights remain the same.

Finally, reaching fixpoints for rich probabilistic abstract domains
appears to require widening \cite{monniaux2000abstract,esparza2011probabilistic} to be cost-effective.  These can be challenging to define and, generally,
lead to a loss in precision.

\paragraph{Probability for Abstract States}

Computing bounds on the probability of a state property has been well-studied.
Di Pierro et al. \cite{di2013probabilistic} develop analyses
to estimate the probability of an abstract state, 
rather than bound it or its probability distribution.  
They formulate their analysis using an abstract 
domain over vector spaces, instead of lattices, and use
the Moore-Penrose pseudo-inverse instead of the usual fixpoint calculation.

Abstract states encode variable domains as matrices, e.g., a 100 by 100 matrix
would be needed to encode the input $x$ for the example in 
Figure~\ref{fig:example}.  While very efficient matrix algorithms
can be employed, the space consumed by this representation can
be significant when scaling to real programs.
Transfer functions operate on these matrices to filter values and
update probabilities along branches and, 
as in Ramalingam's work, weighted sums are used to accumulate probabilities
at control flow merge points.
Di Pierro et al.'s early work was limited to very small 
programs, but more recent
work suggests approaches for abstracting the matrices to significantly
reduce time and space complexity.   

\subsubsection{Handling nondeterminism}
When abstraction of program choice is required or when 
there is no basis for defining an input distribution,
it is natural to use nondeterminism to account for the uncertainty
in program behavior.

In Monniaux's semantics \cite{monniaux2005abstract} choices 
that can be tied to a known
distribution are cleanly separated from those that cannot.
A nondeterministic choice allows for independent outcomes, and
this is modeled by lifting the singleton outcomes of deterministic
semantics to powersets of outcomes.
In the probabilistic setting, the elements of this powerset are
tuples of the abstract domain and the associated weight, defined
above.  So for any nondeterminstic choice, the resulting computation 
is safely modeled by one of these tuples.  The challenge in the
analysis is to select from among those tuples to compute a useful
probability estimate.

More recent work on abstraction in probabilistic data flow
analysis, as well as in model checking, takes a different approach.
In the MDP on the right side of Figure~\ref{fig:pcfgs} the ``$?_i$''
indicate branch outcomes whose probabilities are unknown.
A trace in an MDP can be viewed as the alternation of
probabilistic and nondeterministic choices---whose probabilities are unknown.  
For a given program state, we can formulate a 2-player game
that seeks to compute values for ``$?_i$'' which maximizes (or minimizes)
the probability of reaching that state.
Probabilistic model checkers such as PRISM and PASS 
use this approach to formulate MDP-based analyses.

Consider the reachability of line 5 in the MDP.  There is a single
path consisting of transitions labeled $?_1$, $?_2$, and $0.6$.  
The probability of this path, which is the product of the transition
probabilities, is maximized if $?_1 = \; ?_2 = 1$.
The probability of returning from a call to $m$ without reaching
an assert false requires computing values for $?_1$ and $?_2$ that
maximizes $(?_1 * ?_2 * 0.6) + (?_1 * (1 - ?_2) * 0.3) + ((1 - ?_1) * 0.55)$, 
i.e., the sum of the probabilities along all paths reaching the state.

Abstract interpretation can be applied to the data states in
such approaches~\cite{kwiatkowska2011prism,wachter2010best,esparza2011probabilistic}
to improve efficiency.  These abstractions are, however,
independent of the probabilistic choices implicit in the semantics, and must be specified
by the developer in some way---as in the case of Ramalingam's work.
