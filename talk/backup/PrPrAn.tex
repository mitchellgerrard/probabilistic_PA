\documentclass[xcolor=dvipsnames,10pt]{beamer}
% ********** Style prezentation **********

\usepackage{verbatim}
\usepackage{color}
\usepackage{multimedia}
\usepackage{xmpmulti}
\usepackage[absolute,overlay]{textpos} 
\usepackage{listings}
\usepackage{algorithm}
\usepackage{algorithmic}
\usepackage{amsmath}
\usepackage{pifont}
\usepackage{url}
\usepackage{pict2e}
\usepackage{flushend}
\usepackage{graphicx}
\usepackage{pgf}
\usepackage{url}
\usepackage{tikz}
\usetikzlibrary{calc}
\usetikzlibrary{shapes}
\usetikzlibrary{backgrounds}
\usepackage{pgfplots}
\usepackage{pgfplotstable}

\newcommand{\ignore}[1]{}


\lstset{ %
  language=Java,              
  basicstyle=\footnotesize
}

\pgfdeclarelayer{foreground}
\pgfdeclarelayer{background}
\pgfsetlayers{background,main,foreground}

\mode<presentation>
{
	\usetheme{Warsaw}
}
\definecolor{scarlet}{RGB}{200,0,0}
\setbeamercolor{structure}{fg=scarlet}
\addtobeamertemplate{footline}{
\begin{columns}
  \hfill\includegraphics[height=.75cm]{unl_clear.pdf}\hspace{.5cm}
\end{columns}
\vspace{2mm}
}{\usebeamercolor[bg]{footline}\hfill\raisebox{1mm}[0mm]{\hspace{2mm}}}

\newcommand\x[1]{\ensuremath{\mathit{#1}}}
\newcommand\lrangle[1]{\ensuremath{\langle#1\rangle}}

\def\ON{\ding{52}}
\def\OF{\ding{55}}
\def~{\phantom{0}}

\setbeamertemplate{navigation symbols}{} %remove if navigation symbols are needed
\setbeamersize{text margin left=5mm, text margin right=5mm} % change margin as you wish

\author{Matthew B. Dwyer}

\title[Probabilistic Program Analysis]{Probabilistic Program Analysis}
\subtitle{}


\institute{
Department of Computer Science and Engineering\\
University of Nebraska - Lincoln\\
Lincoln, Nebraska USA\\
}

\date{August 2015}

\begin{document}

\begin{frame}
	\titlepage %displays the title page
\end{frame}


\begin{frame}[fragile]
\frametitle{Program Analysis in a nutshell}
\vfill
\begin{center}
\def\supercircle{(0,0) circle (1.25cm)}
\def\progtraces{(0:3cm) circle (1.0cm)}
\def\subcircle{(0:6cm) circle (0.75cm)}

\tikzset{red/.style={fill=red}}
\tikzset{green/.style={fill=green}}

\begin{tikzpicture}%[scale=1.0]
  \draw \progtraces node[below] {program};
  \draw[red] \supercircle node[yshift=-15mm,below] {over-approximation};
  \draw[green] \subcircle node[yshift=-15mm,below] {under-approximation};
\end{tikzpicture}
\end{center}
\vfill
\end{frame}

\begin{frame}[fragile]
\frametitle{Program Analysis in a nutshell}
\vfill
\begin{center}
\def\supertraces{(0,0) circle (1.0cm)}
\def\supercircle{(0,0) circle (1.25cm)}
\def\subcircle{(0:6cm) circle (0.75cm)}
\def\subtraces{(0:6cm) circle (1.0cm)}

\tikzset{red/.style={fill=red}}
\tikzset{green/.style={fill=green}}

\begin{tikzpicture}%[scale=1.0]
  \begin{scope}
     \clip \supercircle;
     \draw[red, even odd rule] \supercircle \supertraces;
  \end{scope}
  \draw\subtraces;
  \draw[green] \subcircle;
  \draw \supercircle node[yshift=-15mm,below] {extra behavior};
  \draw \subcircle node[yshift=-15mm,below] {missing behavior};
\end{tikzpicture}
\end{center}
\vfill
\end{frame}

\pgfmathdeclarefunction{Gauss}{2}{%
  \pgfmathparse{1/(#2*sqrt(2*pi))*exp(-((\x-#1)^2)/(2*#2^2))}%
}

\pgfmathdeclarefunction{RightMirrorPiledGauss}{2}{%
 \pgfmathparse{x<0?0:2*Gauss(#1,#2)}%
}

\begin{frame}[fragile]
\frametitle{Imagine a normally distributed integer}
\begin{tikzpicture}
\begin{axis}[xlabel=$x$, grid=major, samples=100, domain=-10:10]
\addplot[green,very thick] {Gauss(0,2)};
\end{axis}
\end{tikzpicture}
\end{frame}

\begin{frame}[fragile]
\frametitle{A trivial program}
\vfill
\begin{center}
\begin{lstlisting}
int abs(int x) {
  if (x<0) 
    return -x;
  else 
    return x;
}
\end{lstlisting}
\end{center}
\vfill
\end{frame}

\begin{frame}[fragile]
\frametitle{Here is the output distribution}
\begin{tikzpicture}
\begin{axis}[xlabel=$x$, grid=major, samples=100, domain=-10:10]
\addplot[blue,very thick] {RightMirrorPiledGauss(0,2)};
\end{axis}
\end{tikzpicture}
\end{frame}

\begin{frame}[fragile]
\frametitle{What's going on here?}
\vfill
\begin{tikzpicture}

\begin{axis}[xlabel=$x$, grid=major, samples=100, domain=-10:10]
\addplot[green,very thick] {Gauss(0,2)};
\end{axis}

\begin{pgfonlayer}{background}
\path[fill=black!10] (0,0) rectangle (3.4,5.7);
\node at (2,4) {$x<0$};  
\node at (4.8,4) {$x \ge 0$};  
\end{pgfonlayer}

\end{tikzpicture}
\vfill
Let's think in terms of a very coarse division of the input

\end{frame}

\begin{frame}[fragile]
\frametitle{What's going on here?}
\vfill
\begin{tikzpicture}

\begin{axis}[xlabel=$x$, grid=major, samples=100, domain=-10:10]
\addplot[green,very thick] {Gauss(0,2)};
\end{axis}

\begin{pgfonlayer}{background}
\path[fill=black!10] (0,0) rectangle (3.4,5.7);
\node at (2,4) {$x<0$};  
\node at (4.8,4) {$x \ge 0$};  
\end{pgfonlayer}

\end{tikzpicture}
\vfill
Input values $x \ge 0$ appear on the output unchanged.
\pause

Their mass in the input distribution propagates to the output.

\end{frame}

\begin{frame}[fragile]
\frametitle{What's going on here?}
\vfill
\begin{tikzpicture}

\begin{axis}[xlabel=$x$, grid=major, samples=100, domain=-10:10]
\addplot[green,very thick] {Gauss(0,2)};
\end{axis}

\begin{pgfonlayer}{background}
\path[fill=black!10] (0,0) rectangle (3.4,5.7);
\node at (2,4) {$x<0$};  
\node at (4.8,4) {$x \ge 0$};  
\end{pgfonlayer}

\begin{pgfonlayer}{foreground}
\path[fill] (2.47,1.65) circle [radius=0.1];
\end{pgfonlayer}

\end{tikzpicture}
\vfill
Input values $x < 0$ are transformed.

\end{frame}

\begin{frame}[fragile]
\frametitle{What's going on here?}
\vfill
\begin{tikzpicture}
\begin{axis}[xlabel=$x$, grid=major, samples=100, domain=-10:10]
\addplot[green,very thick] {Gauss(0,2)};
\end{axis}
\begin{pgfonlayer}{background}
\path[fill=black!10] (0,0) rectangle (3.4,5.7);
\node at (2,4) {$x<0$};  
\node at (4.8,4) {$x \ge 0$};  
\end{pgfonlayer}
\begin{pgfonlayer}{foreground}
\path[fill] (2.47,1.65) circle [radius=0.1];
\onslide<2>\draw[->] (2.47,1.65) -> (4.37,1.65);
\end{pgfonlayer}
\end{tikzpicture}
\vfill
Input values $x < 0$ are transformed.  

\onslide<2>Their mass in the input distribution is shifted to $-x$ 

\end{frame}

\begin{frame}[fragile]
\frametitle{What's going on here?}
\vfill
\begin{tikzpicture}
\begin{axis}[xlabel=$x$, grid=major, samples=100, domain=-10:10]
\addplot[green,very thick] {Gauss(0,2)};
\end{axis}
\begin{pgfonlayer}{background}
\path[fill=black!10] (0,0) rectangle (3.4,5.7);
\node at (2,4) {$x<0$};
\node at (4.8,4) {$x \ge 0$};
\end{pgfonlayer}
\begin{pgfonlayer}{foreground}
\path[fill] (4.37,1.65) circle [radius=0.1];
\end{pgfonlayer}
\end{tikzpicture}
\vfill
Input values $x < 0$ are transformed.  

Their mass in the input distribution is shifted to $-x$ 

and accumulates in the output distribution
\end{frame}

\begin{frame}[fragile]
\frametitle{What's going on here?}
\vfill
\begin{tikzpicture}
\begin{axis}[xlabel=$x$, grid=major, samples=100, domain=-10:10]
\addplot[green,very thick] {Gauss(0,2)};
\end{axis}
\begin{pgfonlayer}{background}
\path[fill=black!10] (0,0) rectangle (3.4,5.7);
\node at (2,4) {$x<0$};
\node at (4.8,4) {$x \ge 0$};
\end{pgfonlayer}
\begin{pgfonlayer}{foreground}
\path[fill] (4.37,2.85) circle [radius=0.1];
\end{pgfonlayer}
\end{tikzpicture}
\vfill
Input values $x < 0$ are transformed.  

Their mass in the input distribution is shifted to $-x$ 

and accumulates in the output distribution
\end{frame}

\begin{frame}[fragile]
\frametitle{Here is the output distribution}
\begin{tikzpicture}
\begin{axis}[xlabel=$x$, grid=major, samples=100, domain=-10:10]
\addplot[blue,very thick] {RightMirrorPiledGauss(0,2)};
\end{axis}

\begin{pgfonlayer}{foreground}
\path[fill] (4.37,1.67) circle [radius=0.1];
\end{pgfonlayer}

\end{tikzpicture}
\end{frame}

\begin{frame}[fragile]
\frametitle{Bounding distribution}
\begin{tikzpicture}
\begin{axis}[xlabel=$x$, grid=major, samples=100, domain=-10:10]
\addplot[green,very thick] {Gauss(0,2)};
\end{axis}
\begin{pgfonlayer}{foreground}
% distance divided by 10 is 5.71 = 6.28 - 0.57
\path[draw]  (0.2845, 0.5)          rectangle (0.2845+0.1*5.71, 0.6);
\path[draw]  (0.2845+0.1*5.71, 0.5) rectangle (0.2845+0.2*5.71, 0.6);
\path[draw]  (0.2845+0.2*5.71, 0.5) rectangle (0.2845+0.3*5.71, 0.8);
\path[draw]  (0.2845+0.3*5.71, 0.5) rectangle (0.2845+0.4*5.71, 2.0);
\path[draw]  (0.2845+0.4*5.71, 0.5) rectangle (0.2845+0.5*5.71, 4.9);
\path[draw]  (0.2845+0.5*5.71, 0.5) rectangle (0.2845+0.6*5.71, 5.3);
\path[draw]  (0.2845+0.6*5.71, 0.5) rectangle (0.2845+0.7*5.71, 4.9);
\path[draw]  (0.2845+0.7*5.71, 0.5) rectangle (0.2845+0.8*5.71, 2.0);
\path[draw]  (0.2845+0.8*5.71, 0.5) rectangle (0.2845+0.9*5.71, 0.8);
\path[draw]  (0.2845+0.9*5.71, 0.5) rectangle (0.2845+1.0*5.71, 0.6);
\path[draw]  (0.2845+1.0*5.71, 0.5) rectangle (0.2845+1.1*5.71, 0.6);

\pause
\node at (0.22+0.15*5.71, 0.8) {\footnotesize $0.003$};
\node at (0.22+0.25*5.71, 1.0) {\footnotesize $0.01$};
\node at (0.22+0.35*5.71, 2.2) {\footnotesize $0.065$};
\node at (0.22+0.45*5.71, 5.1) {\footnotesize $0.185$};
\node at (0.22+0.55*5.71, 5.5) {\footnotesize $0.21$};
\end{pgfonlayer}
\end{tikzpicture}
\end{frame}

\begin{frame}[fragile]
\frametitle{Bounding $Pr([-1,1])$}
\begin{tikzpicture}
\begin{axis}[xlabel=$x$, grid=major, samples=100, domain=-10:10]
\addplot[green,very thick] {Gauss(0,2)};
\end{axis}
\begin{pgfonlayer}{foreground}
% distance divided by 10 is 5.71 = 6.28 - 0.57
\path[draw]  (0.2845, 0.5)          rectangle (0.2845+0.1*5.71, 0.6);
\path[draw]  (0.2845+0.1*5.71, 0.5) rectangle (0.2845+0.2*5.71, 0.6);
\path[draw]  (0.2845+0.2*5.71, 0.5) rectangle (0.2845+0.3*5.71, 0.8);
\path[draw]  (0.2845+0.3*5.71, 0.5) rectangle (0.2845+0.4*5.71, 2.0);
\path[draw]  (0.2845+0.4*5.71, 0.5) rectangle (0.2845+0.5*5.71, 4.9);
\path[draw]  (0.2845+0.5*5.71, 0.5) rectangle (0.2845+0.6*5.71, 5.3);
\path[draw]  (0.2845+0.6*5.71, 0.5) rectangle (0.2845+0.7*5.71, 4.9);
\path[draw]  (0.2845+0.7*5.71, 0.5) rectangle (0.2845+0.8*5.71, 2.0);
\path[draw]  (0.2845+0.8*5.71, 0.5) rectangle (0.2845+0.9*5.71, 0.8);
\path[draw]  (0.2845+0.9*5.71, 0.5) rectangle (0.2845+1.0*5.71, 0.6);
\path[draw]  (0.2845+1.0*5.71, 0.5) rectangle (0.2845+1.1*5.71, 0.6);

\pause
\draw[<->,very thick]  (0.2845+0.5*5.71, 0.2) -> (0.2845+0.6*5.71, 0.2);
\end{pgfonlayer}
\end{tikzpicture}
\vfill
\pause
How many elements are in the domain? \pause 3
\pause

What is the mass of each element? \pause $\le 0.21$ 
\pause

$Pr([-1,1]) \le 0.63 = 3*0.21$
\end{frame}

\begin{frame}[fragile]
\frametitle{Bounding $Pr([0,5])$}
\begin{tikzpicture}
\begin{axis}[xlabel=$x$, grid=major, samples=100, domain=-10:10]
\addplot[green,very thick] {Gauss(0,2)};
\end{axis}
\begin{pgfonlayer}{foreground}
% distance divided by 10 is 5.71 = 6.28 - 0.57
\path[draw]  (0.2845, 0.5)          rectangle (0.2845+0.1*5.71, 0.6);
\path[draw]  (0.2845+0.1*5.71, 0.5) rectangle (0.2845+0.2*5.71, 0.6);
\path[draw]  (0.2845+0.2*5.71, 0.5) rectangle (0.2845+0.3*5.71, 0.8);
\path[draw]  (0.2845+0.3*5.71, 0.5) rectangle (0.2845+0.4*5.71, 2.0);
\path[draw]  (0.2845+0.4*5.71, 0.5) rectangle (0.2845+0.5*5.71, 4.9);
\path[draw]  (0.2845+0.5*5.71, 0.5) rectangle (0.2845+0.6*5.71, 5.3);
\path[draw]  (0.2845+0.6*5.71, 0.5) rectangle (0.2845+0.7*5.71, 4.9);
\path[draw]  (0.2845+0.7*5.71, 0.5) rectangle (0.2845+0.8*5.71, 2.0);
\path[draw]  (0.2845+0.8*5.71, 0.5) rectangle (0.2845+0.9*5.71, 0.8);
\path[draw]  (0.2845+0.9*5.71, 0.5) rectangle (0.2845+1.0*5.71, 0.6);
\path[draw]  (0.2845+1.0*5.71, 0.5) rectangle (0.2845+1.1*5.71, 0.6);

\pause
\draw[<->,very thick]  (0.2845+0.55*5.71, 0.2) -> (0.2845+0.8*5.71, 0.2);
\end{pgfonlayer}
\end{tikzpicture}
\vfill
\pause
How many elements are in the domain? \pause 5

\pause
What is the mass of each element? \pause $\le 0.21, \le 0.185, \le 0.065$

\pause
$Pr([0,5]) \le 0.92 = 2*0.21 + 2*0.185 + 2*0.065$
\end{frame}



\end{document}
