\documentclass[]{article}

\usepackage{amssymb}
\setcounter{tocdepth}{3}
\usepackage{graphicx}
\usepackage{color}
\usepackage{amsmath,amssymb,mathtools}
\usepackage{listings}
\usepackage{algorithm}
\usepackage{algorithmic}
\usepackage{centernot}
\usepackage{tikz}
\usetikzlibrary{calc}
\usetikzlibrary{shapes}
\usepackage{pgfplots}
\usepackage{pgfplotstable}


\newcommand\x[1]{\ensuremath{\mathit{#1}}}
\newcommand\lrangle[1]{\ensuremath{\langle#1\rangle}}

\makeatletter
\providecommand{\bigsqcap}{%
  \mathop{%
    \mathpalette\@updown\bigsqcup
  }%
}
\newcommand*{\@updown}[2]{%
  \rotatebox[origin=c]{180}{$\m@th#1#2$}%
}
\makeatother


%\usepackage{url}
%\urldef{\mailsa}\path|{dwyer, mgerrard}@cse.unl.edu|    
%\newcommand{\keywords}[1]{\par\addvspace\baselineskip
%\noindent\keywordname\enspace\ignorespaces#1}


% comment macro
\newcommand{\mycomment}[1]{\textit{\textcolor{red}{#1}}}
\newcommand{\ignore}[1]{}

\title{Distribution Transformer}

\begin{document}

\pgfmathdeclarefunction{Gauss}{2}{%
  \pgfmathparse{1/(#2*sqrt(2*pi))*exp(-((\x-#1)^2)/(2*#2^2))}%
}

\pgfmathdeclarefunction{RightMirrorPiledGauss}{2}{%
 \pgfmathparse{x<0?0:2*Gauss(#1,#2)}%
}

\begin{figure}
\begin{tikzpicture}
\begin{axis}[xlabel=$x$, grid=major, samples=100, domain=-10:10]
\addplot[solid] {Gauss(0,2)};
\addplot[dashed] {RightMirrorPiledGauss(0,2)};
\end{axis}
\end{tikzpicture}
\caption{Input Distribution $N(0,2)$ (solid), Output Distribution (dashed)}
\label{fig:input}
\end{figure}

Consider the simple program that accepts an integer as
input and returns its absolute value.
\begin{lstlisting}[language=Java,basicstyle=\scriptsize\ttfamily]
int abs(int x) {
  if (x<0) 
    return -x;
  else 
    return x;
}
\end{lstlisting}
Figure~\ref{fig:input} shows the normally distributed input (in solid)
and the output distribution which accumulates the probability mass 
of the negative inputs onto the negated positive values on output.


\end{document}
