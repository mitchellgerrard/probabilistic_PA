\documentclass[xcolor=dvipsnames,10pt]{beamer}
% ********** Style prezentation **********

\usepackage{verbatim}
\usepackage{color}
\usepackage{multimedia}
\usepackage{xmpmulti}
\usepackage[absolute,overlay]{textpos} 
\usepackage{listings}
\usepackage{algorithm}
\usepackage{algorithmic}
\usepackage{amsmath}
\usepackage{pifont}
\usepackage{url}
\usepackage{pict2e}
\usepackage{flushend}
\usepackage{graphicx}
\usepackage{pgf}
\usepackage{url}
\usepackage{tikz}
\usetikzlibrary{calc}
\usetikzlibrary{shapes}
\usetikzlibrary{backgrounds}
\usepackage{pgfplots}
\usepackage{pgfplotstable}

\newcommand{\ignore}[1]{}


\lstset{ %
  language=Java,              
  basicstyle=\footnotesize
}

\mode<presentation>
{
	\usetheme{Warsaw}
}
\definecolor{scarlet}{RGB}{200,0,0}
\setbeamercolor{structure}{fg=scarlet}
\addtobeamertemplate{footline}{
\begin{columns}
  \hfill\includegraphics[height=.75cm]{unl_clear.pdf}\hspace{.5cm}
\end{columns}
\vspace{2mm}
}{\usebeamercolor[bg]{footline}\hfill\raisebox{1mm}[0mm]{\hspace{2mm}}}

\newcommand\x[1]{\ensuremath{\mathit{#1}}}
\newcommand\lrangle[1]{\ensuremath{\langle#1\rangle}}

\def\ON{\ding{52}}
\def\OF{\ding{55}}
\def~{\phantom{0}}

\setbeamertemplate{navigation symbols}{} %remove if navigation symbols are needed
\setbeamersize{text margin left=5mm, text margin right=5mm} % change margin as you wish

\author{Matthew B. Dwyer}

\title[Probabilistic Program Analysis]{Probabilistic Program Analysis}
\subtitle{GTTSE part 4 : Don't have a PhD topic? Here take one of mine!}


\institute{
Department of Computer Science and Engineering\\
University of Nebraska - Lincoln\\
Lincoln, Nebraska USA\\
}

\date{August 2015}

\begin{document}

\begin{frame}
	\titlepage %displays the title page
\end{frame}

\begin{frame}{Explore quantitative program analyses}
\vfill
In security researchers moved from information flow to \textit{quantitative}
information flow in an attempt to characterize the severity of vulnerabilities.
\vfill
There may be additional opportunities for adapting classic analyses 
to produce quantitative information that is not 
not necessarily probabilistic.
\vfill
Quantification can shift questions from possibility/impossibility to how many,
i.e., channel capacity.
\vfill
For example, dependence analysis could be modified to compute edge
weights that permit the capacity of dependence chains to be computed?
\vfill
This would permit a ``capacity threshold'' notion of slicing to
be performed.
\vfill
Would this be useful?  \pause  You tell us in a few years.
\vfill
\end{frame}

\begin{frame}{Rank errors based on execution probability}
\vfill
A number of analysis techniques, e.g., model checking, provide
error reports in the form of counter examples, i.e., a trace.
\vfill
There can be many such traces and determining which one to look
at can be a challenge.
\vfill
Lots of work on error ranking in the mid-2000s, but
probabilistic technique might permit a new approach where 
the probability of taking the path that the counter example
took could allow developers to focus on ``more likely'' errors first.
\vfill
\end{frame}

\begin{frame}{Customizing supporting techniques for program analysis}
\vfill
Model counting is a general problem; the complexity of \#SAT has been
studied for many decades.
\vfill
SAT techniques became fast by focusing 
on the cases that arose in practice.
\vfill
This led to a proliferation of \textit{theory fragments} with associated
efficient decision procedures.
\vfill
\pause
The same approach seems possible for \# procedures.
\vfill
Simple example: polyhedra that are regular hyperrectangles are trivial to count, i.e., $\Pi_{e \in Edge} length(e)$
\vfill
A first step in this direction would be to characterize the types
of queries that arise in program analysis which require \#.
\vfill
Collecting a large corpus of such queries and classifying them 
would be very useful in providing a ``target'' for \# researchers.
\vfill
\end{frame}

\begin{frame}{DSL Design for Analysis}
\vfill
DSLs are being designed for many domains, e.g., 
robotics, ecological/biological systems, in which 
stochasticity is fundamental.
\vfill
How can the design of these DSLs be set up to 
facilitate probabilistic analysis?
\vfill
Can we incorporate notions of input distribution into the
language?   For instance as an assumption about inputs that
can be exploited for analysis and checked at runtime? 
\vfill
Are there notions of probabilistic contract that make sense
in this setting that could permit modular analysis?
\vfill
\end{frame}

\begin{frame}{Moving from explicit to implicit branch probabilities}
\vfill
Popular probabilistic model checking tools like 
like PRISM and PASS still require that probabilities annotate
transitions in the model.
\vfill
In some cases this is easy, e.g., modeling \texttt{bernouli(0.5)}, but
in others it is not, e.g., when it is based in complex ways on 
input data.
\vfill
Is it possible to incorporate the tracking of symbolic constraints
reaching a branch and then using \# techniques to eliminate some
of the explicit probabilities? 
\vfill
This is tricky because you need to shift from a fully path-sensitive
setting, like symbolic execution, to one where paths are merged, but
it is possible.
\vfill
At the very least it seems possible to run a separate analysis that
attempts to determine the accuracy of those explicit probabilities?
\vfill
\end{frame}

\begin{frame}{Residual probabilistic program analysis}
\vfill
In a residual analysis one stages a series of analyses.
\vfill
An analysis does what it can on the program and then passes
the unanalyzed part, the residue, on to the next one in the series.
\vfill
What if you were to run a probabilistic symbolic execution 
for whatever resource budget you can tolerate and then target
the ``unanalyzed'' portion of the program using a data flow
analysis?
\vfill
Alternatively you run probabilistic data flow first, then target
the regions of the program for which you have imprecise probability
information for refinement using symbolic execution.
\vfill
One could use slicing techniques to extract that residue.
\vfill
This would combine the precision of the symbolic execution 
with the scalability of data flow.
\vfill
\end{frame}

\begin{frame}{Probabilistic program differencing}
\vfill
Probabilistic symbolic execution is particularly well suited for quantifying the difference between two versions of a program~\cite{Filieri2015b}. 
\vfill
This means that it could be used to rank how close a program is to a given oracle program.
\vfill
This has applications in mutation analysis, program repair, approximate computing or even in marking student assignments.
\vfill
\end{frame}

\begin{frame}{Symbolically executing ``probabilistic programs''}
\vfill
Probabilistic programming is becoming very popular.
\vfill
Most current approaches rely on sampling.
\vfill
Statistical symbolic execution can be more efficient because 
each sample accumulates the \texttt{mass} of the sampled path
instead of the mass of the single sampled trace.
\vfill
Extending symbolic execution to support these programs requires support
for conditioning, i.e., the \texttt{observe(...)} statement.
\vfill
From a path generation perspective this is trivial, i.e., it is equivalent
to support of \texttt{assume(...)}.
\vfill
It is trickier to handle the renormalization of the output distribution; but this seems possible.
\vfill
Handling incompleteness, e.g., when the probabilistic program iterates over
large domains, remains an open challenge.
\vfill
\end{frame}

\begin{frame}{Final words}
\vfill
Send us your ideas for improving the briefing paper!
\vfill
We want it to be a useful resource for the community.
\vfill
\end{frame}

\end{document}
